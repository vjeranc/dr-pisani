\citet*{daume15lols} uvode terminologiju \textit{rollin} i \textit{rollout} u
kontekst metoda učenja pretraživanja. Termin \textit{rollout} originalno je
naziv za tehniku analize pozicija i poteza u igri
\textit{backgammon}.\footnote{\url{https://en.wikipedia.org/wiki/Rollout_(backgammon)}}
Cilj je dovoljan broj puta krenuti od iste pozicije i odigravanjem (bacajući
kocku) dolaziti do kraja igre i bilježiti rezultat svakog ishoda. Broj pobjeda i
poraza daje procjenu o dobroti pozicije. U okviru podržanog učenja tehnika se
koristi tako da se iz nekog stanja nasumično odabiru sljedeći potezi -- odabir
može biti baziran na do tada naučenoj politici. Nakon što se izvrši jedan niz poteza i
dobije rezultat, cijeli postupak se ponavlja počinjanjem iz istog stanja tako da
je procjena o dobroti pozicije što bolja.

Termin \textit{rollin} nije bio prisutan kod podržanog učenja jer postoji samo
naučena politika, ali se podrazumijeva da se do željenog stanja dolazi koristeći
naučenu politiku -- ako su neka stanja nedostižna trenutnoj naučenoj politici, onda
bi bilo bespotrebno vrednovati odluke iz stanja do kojih se ne može doći. U
kontekstu metoda učenja pretraživanja \textit{rollin} je postupak dolaska do
određenog stanja, a kako je na raspolaganju naučena i referentna politika dobro je
vidjeti koja strategija vodi do brzog i konzistentnog učenja -- hoće li se
dolaziti do budućeg stanja referentnom ili naučenom politikom, a možda za svaki
korak odabrati nekom vjerojatnošću jednu ili drugu. Na slici \ref{fig:rollinout}
prikazan je postupak korištenja politika u postupku \textit{rollin} i \textit{rollout}
za vrednovanje odluka gdje su odluke predstavljene bridovima, a stanja čvorovima.

\begin{figure}
\centering
\begin{tikzpicture}[
  >={Triangle[]},
  b/.style = {lightblue,fill=lightblue!40},
  g/.style = {darkgreen,fill=darkgreen!40}
  ]
  \matrix[
  matrix of nodes,column sep=1cm,row sep=.1cm,
  nodes={
    lightgray,draw,thick,fill=lightgray!40,circle,
    minimum size=3ex, inner sep=1pt,anchor=south
  }] (m) {
          &       &       &       &     {}&     {}\\
          &       &       &       &     {}&     {}\\
          &     {}&     {}&|[g]|{}&|[g]|{}&|[g]|Z \\
          &       &       &       &     {}&     {}\\
   |[b]|P &|[b]|{}&|[b]|R &|[b]|{}&|[b]|{}&|[b]|Z \\
          &       &       &       &     {}&     {}\\
          &     {}&     {}&|[g]|{}&|[g]|Z &       \\
          &       &       &       &     {}&       \\
          &       &       &       &     {}&       \\
  };
    % Green arrows
  \draw[darkgreen,->, line width=1.2pt] (m-5-3.east) to[bend left] (m-3-4.west);
  \draw[darkgreen,->, line width=1.2pt] (m-5-3.east) to[bend right] (m-7-4.west);
  % Gray arrows
  \foreach \i[evaluate=\i as \j using int(\i+1)] in {1,2} {
    \foreach \row/\bend in {3/left, 7/right}
      \draw[lightgray,->, line width=1.2pt] (m-5-\i.east) to[bend \bend]  (m-\row-\j.west);
  }
  \foreach \i[evaluate=\i as \j using int(\i+1)] in {4,5} {
    \foreach \row/\bend in {1/left, 2/left}
      \draw[lightgray,->, line width=1.2pt] (m-3-\i.east) to[bend \bend]  (m-\row-\j.west);
  }
  \foreach \i[evaluate=\i as \j using int(\i+1)] in {4,5} {
    \foreach \row/\bend in {4/left, 6/right}
      \draw[lightgray,->, line width=1.2pt] (m-5-\i.east) to[bend \bend]  (m-\row-\j.west);
  }
  \foreach \row/\bend in {8/right, 9/right}
    \draw[lightgray,->, line width=1.2pt] (m-7-4.east) to[bend \bend]  (m-\row-5.west);
  % Black arrows
  \foreach \i [remember=\i as \lasti (initially 4)] in {5,6}
    \draw[->, line width=1.2pt] (m-3-\lasti.east) to (m-3-\i.west);
  \foreach \i [remember=\i as \lasti (initially 1)] in {2,...,6}
    \draw[->, line width=1.2pt] (m-5-\lasti.east) to (m-5-\i.west);
  \draw[->, line width=1.2pt] (m-7-4.east) to (m-7-5.west);
  % Loss
  \node[right=of m-3-6] (loss1) {$y_z \in \mathcal{Y}, l(y_z) = 0$};
  \node[right=of m-5-6] (loss2) {$y_z \in \mathcal{Y}, l(y_z) = 1$};
  \node[right=of m-7-5] (loss3) {$y_z \in \mathcal{Y}, l(y_z) = 2$};
  \draw[->, line width=1.2pt] (m-3-6.east) to (loss1);
  \draw[->, line width=1.2pt] (m-5-6.east) to (loss2);
  \draw[->, line width=1.2pt] (m-7-5.east) to (loss3);
  % Braces
  \draw[decorate,decoration={brace,amplitude=10pt},black,thick]
    (m-7-3 |- m-7-3.south) -- node[below=10pt] (rollin) {\textit{rollin}} (m-5-1 |- m-7-3.south);
  \draw[decorate,decoration={brace,amplitude=10pt},black,thick]
    (m-5-6 |- m-9-5.south) -- node[below=10pt] (rollout) {\textit{rollout}} (m-5-4 |- m-9-5.south);
  \path (rollin) -- node[black,align=right,rotate=90] {odstupanja od \\ jednog koraka} (rollout);

  \node[shape=rectangle,draw=black,thick, above=of m-5-1] (xinX) {$x \in \mathcal{X}$};
  \draw[->, line width=1.2pt,out=-90,in=120] (xinX) to (m-5-1);
\end{tikzpicture}
\caption[Prikaz postupka \textit{rollin} i \textit{rollout} kod učenja
pretraživanja.]{Prikaz postupka \textit{rollin} i \textit{rollout} kod učenja
pretraživanja. Ulaz $x \in \mathcal{X}$ vodi pretragu svojim značajkama. Kreće
se iz stanja $P$ i dva se puta odabire srednja odluka od moguće tri koristeći
odabranu politiku za \textit{rollin}. Sivi čvorovi se ne pretražuju. U koraku $R$ algoritam
učenja odabire sve moguće odluke radeći odstupanje od jednog koraka i
primjenjuje odabranu politiku u postupku \textit{rollout} za svaku moguću odluku dovoljno puta dok ne
dođe do kraja. Na kraju pomoću potpunog niza odluka $y_z \in \mathcal{Y}$ dobiva
se informacija o gubitku i ona se koristi kao procjena dobrote svake odluke.
Nakon postupka moguće je zaključiti da se gornjom odlukom koja odstupa od
odabranog srednjeg stanja može gubitak smanjiti za 1. Primjer preuzet iz
\citep[str.~3]{daume15lols}}
\label{fig:rollinout}
\end{figure}

\cite{daume15lols} pokazuju da korištenje višerazrednog klasifikatora
osjetljivog na trošak, koji ima dobre ograde žaljenja, ne garantira konzistentnu
naučenu politiku za problem strukturnog predviđanja. Ovisno o načinu na koji se
radi \textit{rollin} i \textit{rollout} i pretpostavci o tome kakva je
referentna politika, moguće je dobiti nekonzistentnu redukciju. Rezultat je
priložen na slici \ref{fig:policyresult}. Ovisno o tome koje se politike koriste
u postupcima \textit{rollin} ili \textit{rollout}, moguće je dobiti naučenu
politiku koja nije konzistentna (ima veliko strukturno žaljenje) ili politiku
koja nije lokalno optimalna. U slučaju korištenja naučene politike za
\textit{rollin} i \textit{rollout}, problem strukturnog predviđanja reducira se
na problem podržanog učenja koji je puno teži jer se za učenje uopće ne koristi
znanje referentne politike. Jedini dobar pristup kojim se postiže lokalna
optimalnost i konzistentna redukcija je onaj u kojem se koristi mješovita
politika za \textit{rollout} -- opis metode lokalno optimalnog učenja
pretraživanja nalazi se u potpoglavlju \ref{ch:LOLS}. Ako se pretpostavi da je
referentna politika optimalna i to je stvarno slučaj, onda će učenje koristeći
\textit{rollin} s naučenom, a \textit{rollout} s referentnom rezultirati s
konzistentnom i lokalno optimalnom naučenom politikom. Za mnoštvo problema nije
lako definirati optimalnu politiku i zanimljiv je rezultat da upravo
\textit{rollout} s mješovitom politikom ima garancije da će, ako je u prostoru
odluka moguće raditi lokalno uspinjanje na vrh \engl{local hill climbing},
naučena politika biti konstantno lošija od dane referentne politike ili će biti
bolja od referentne politike. Takvu garanciju pristupi prije algoritma
\textsc{LOLS} nisu imali. Problem je postojao čak i kod metode učenja
pretraživanja \textsc{Searn}. Navedena karakteristika nije pretjerano korisna
ako je referentna politika loša, u tom slučaju učenje se odvija kao da referenta
politika ne postoji (informacija koju ona daje je beskorisna). Ako je lako
raditi uspinjanje na vrh u prostoru odluka, onda bi na problemu jednako dobro
radilo i podržano učenje.

% If you use beamer only pass "xcolor=table" option, i.e. \documentclass[xcolor=table]{beamer}
\begin{figure}
\centering
\begin{tabular}{|
>{\columncolor[HTML]{FFFFC7}}l |
>{\columncolor[HTML]{C0C0C0}}c |
>{\columncolor[HTML]{C0C0C0}}c |
>{\columncolor[HTML]{C0C0C0}}c |}
\hline
\multicolumn{1}{|c|}{\cellcolor[HTML]{C0C0C0}Rollout $\rightarrow$} & \cellcolor[HTML]{C0C0C0}                                     & \cellcolor[HTML]{C0C0C0}                                     & \cellcolor[HTML]{C0C0C0}                                   \\
\multicolumn{1}{|c|}{\cellcolor[HTML]{FFFFC7}$\downarrow$ Rollin}   & \multirow{-2}{*}{\cellcolor[HTML]{C0C0C0}\textbf{Referentna}} & \multirow{-2}{*}{\cellcolor[HTML]{C0C0C0}\textbf{Mješovita}} & \multirow{-2}{*}{\cellcolor[HTML]{C0C0C0}\textbf{Naučena}} \\ \hline
\textbf{Referentna}                                                  & \multicolumn{3}{c|}{\cellcolor[HTML]{FFCCC9}Nekonzistentna redukcija}                                                                                                                    \\ \hline
\textbf{Naučena}                                             & \cellcolor[HTML]{FFCCC9}Nije lokalno optimalna               & \cellcolor[HTML]{C5F7C5}Dobra                                & \cellcolor[HTML]{FFCCC9}Podržano učenje                    \\ \hline
\end{tabular}
\caption[Rezultati teoretske analize načina učenja koristeći postupke
\textit{rollin} i \textit{rollout}.]{Rezultati teoretske analize načina učenja
koristeći postupke \textit{rollin} i \textit{rollout}. Ovisno o odabiru
\textit{rollin} i \textit{rollout} para naučena politika može biti
nekonzistentna ili bez svojstva lokalne optimalnosti, problem učenja politike se
može svesti na podržano učenje ili naučena politika ipak može biti konzistentna
i lokalno optimalna. Mješovita politika je politika kod koje se nekom
vjerojatnošću bira referentna ili naučena. Tablica preuzeta iz
\citep[str.~4]{daume15lols}}
\label{fig:policyresult}
\end{figure}

Da bi se izbjeglo ponavljanje nekih značajki, prednosti i ciljeva algoritama
učenja pretraživanja njihove karakteristike detaljnije su analizirane u zasebnim
potpoglavljima (\ref{ch:metodesearchlearn}).
