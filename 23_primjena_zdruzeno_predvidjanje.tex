Najmoćnija primjena metoda učenja pretraživanja je združeno predviđanje i učenje
\engl{joint prediction and learning}. Zbog činjenice da je moguće koristiti bilo
koju funkciju gubitka moguće je kombinirati više zadataka istovremeno. Trenutni
označivači vrste riječi su postigli točnost od otprilike 97-98\%
\citep{manning2011part} i ne mogu više od toga. Jedan od razloga je taj što neke
oznake nije lako odrediti bez informacije o sintaksi rečenice. U \lts{} okviru
procjenu troška pojedine oznake vrste riječi moguće je proširiti tako da se
nakon označene cijele rečenice izvrši ovisnosno parsiranje. Gubitak koji je
dobiven nakon parsiranja proslijedit će informaciju i poboljšati procjenu
troška za pojedine oznake vrste riječi. Primjena združenog predviđanja na ovom
konkretnom zadatku postoji, ali brzina predviđanja i učenja nije ni približno
zadovoljavajuća \citep{bohnet2012transition}.

Problem prepoznavanja imenovanih entiteta i izlučivanje relacija između entiteta
u tekstu, dva su zadatka koja bi bilo dobro raditi združeno. Očito učinkovitost
izlučivanja relacija ovisi o tome koliko dobro su entiteti prepoznati.
Propagiranje gubitka zadatka izlučivanja relacija bi trebalo dodatno poboljšati
prepoznavanje entiteta. U \lts{} okviru moguće je, ako su problemi slijedno
povezani (prvo se izvršava jedan, a nakon njega drugi), učiti model na jednom
zadatku, a nakon toga učiti na oba istovremeno. Tako se može izbjeći potreba za
ogromnom količinom označenih podataka koji imaju oznake za oba zadatka. Ako se
uči prepoznavanje imenovanih entiteta na skupu koji nema označene relacije, onda
je prisutna lošija procjena troška pojedine odluke -- referentna polica nije
optimalna, ali nije ni jako loša. Učenje na skupu podataka sa svim oznakama bi
uvelo optimalnu policu koja ne bi počinjala od nule za zadatak prepoznavanja
imenovanih entiteta te bi onda bilo moguće popraviti krive procjene novim
podacima na zadatku prepoznavanja entiteta.

Zbog činjenice da je za združeno predviđanje u \lts{} okviru potrebna samo
združena procjena troška, za svaki problem može se koristiti zaseban model. Time
je izbjegnuto korištenje značajki za probleme koji od tih značajki ne bi imali
koristi. Ako se problemi rješavaju slijedno, kao što je slučaj u označavanju
vrste riječi s ovisnosnim parsiranjem, onda je moguće stati sa zaključivanjem
nakon izvršenog prvog zadatka. To omogućuje efikasno korištenje modela
koji su združeni bez da se testiranje vrši združeno. Ovako nešto je jako rijetko
u literaturi i mnoštvo modela koje koristi algoritme dinamičkog programiranja
mora raditi \textit{backtracking}. Bez potpune enumeracije ne postoji sigurnost
da je pronađen najvjerojatniji niz odluka za prvi zadatak stoga je potrebno
dohvatiti sve optimalne odluke za združeni zadatak. Ovisnosno parsiranje
zahtijeva puno više značajki i najbolji parseri ne mogu parsirati više od
nekoliko tisuća riječi po sekundi -- \citet{andor2016globally} opisuju parser
koji postiže najbolje rezultate, ali mu je brzina oko 600 riječi po sekundi na
osobnom računalu unatoč tome što je složenost algoritma koji je u pozadini
$O(T)$, gdje je $T$ duljina rečenice. Združeno učenje bi omogućilo bolje
označavanje vrste riječi, ali bi očuvalo brzinu na tom zadatku koja je za sustav
Vowpal Wabbit preko nekoliko stotina tisuća znački u sekundi \citep{daume14lts}.
