\citet{daume06thesis} u svojoj doktorskoj disertaciji daje doprinos otkrićem
prve metode učenja pretraživanja \textsc{Searn} \engla{search + learn}{Searn}.
Metodu primjenjuje na mnoštvo problema u području obrade prirodnog jezika:
\begin{inlinelist}
  \item označavanje vrste riječi,
  \item prepoznavanje imenovanih entiteta,
  \item plitko parsiranje \engl{shallow parsing},
  \item združeno označavanje vrste riječi i plitko parsiranje,
  \item prepoznavanje i praćenje entiteta (uključuje razrješavanje koreferenci),
  \item sažimanje teksta koristeći više dokumenata i
  \item poravnavanje teksta na engleskom i španjolskom koristeći latentne
  varijable \engl{text alignment with hidden variables}.
\end{inlinelist}
Predložio je primjenu na strojno prevođenje i ovisnosno parsiranje koja je u
okviru učenja pretraživanja nedavno ostvarena
\citep{he2015syntax,chang2015learning}.

\begin{algorithm}
\caption{Učenje + Pretraživanje (\textsc{Searn})}\label{alg:searn}
\begin{algorithmic}[1]
\Require Skup podataka $\{x_i, y_i\}_{i=1}^N$ uzet iz distribucije $\mathcal{D}$,
  $\beta \geq 0$. %% -- parametar mješavine polica.
\State Inicijaliziraj policu $\hat{\pi}_0 = \pi\ssymbol{1}$.
\ForAll{$I \in \big[0,1,2,\ldots,P)$}
  \ForAll{$i \in \{1,2,\ldots,N\}$}
    \State Inicijaliziraj $\Gamma = \emptyset$. \Comment{skup primjera osjetljivih na trošak.}
    \ForAll{$t \in \{0,1,2,\ldots,T_i-1\}$}
      \State Primijeniti $t$ puta policu $\pi^{\text{in}} = \hat{\pi}_{I}$  i stići do $s_t$. \Comment{Rollin.}
      \ForAll{$a \in A(s_t)$}\label{alg:searn:action}
        \State Neka je  $\pi^{\text{out}} = \hat{\pi}_{I}$.\label{alg:searn:rolloutpolicy}
        \State Procjeni trošak $c_{i,t}(a)$ koristeći $T-t-1$ puta policu $\pi^{\text{out}}$. \Comment{Rollout.}\label{alg:searn:rollout}
      \EndFor
      \State Generiraj vektor značajki $\mathbf{\Phi}(x_i, s_t)$.
      \State Postavi $\Gamma = \Gamma \cup \{\langle c_{i,t}, \mathbf{\Phi}(x_i, s_t) \rangle\}$.
    \EndFor
  \EndFor
  \State $\hat{\pi}' \gets \textsc{Train}(\Gamma)$.\label{alg:searn:train}
  \State $\hat{\pi}_{I+1} \gets \beta \hat{\pi}' + (1-\beta) \hat{\pi}_{I}$. \label{alg:searn:mixture}
\EndFor
\State \Return $\hat{\pi}_{P}$ bez $\pi\ssymbol{1}$
\end{algorithmic}
\end{algorithm}

Pseudokod \ref{alg:searn} prikazuje \textsc{Searn} algoritam. Algoritam radi
više prolaza preko skupa za učenje određen brojem $P$. Za svaki primjer $\{x_i,
y_i\}$ dolazi do nekog stanja $s_t$ koristeći \textit{rollin} policu. Ona je pri
prvom prolazu jednaka referentnoj polici, a u svim ostalima ta polica je
stohastička mješavina između polica naučenih u svakom prolazu i referentne
police -- rezultat primjene \textit{rollin} police je kod problema označavanja
vrste riječi niz oznaka, a stanje $s_t$ do kojeg smo stigli je riječ na mjestu
$t$ koju još nismo označili. Linija \ref{alg:searn:mixture} pokazuje izračun
nove police. Izračun ne radi interpolaciju naučenih težina nego samo prilagođava
vjerojatnosti odabira do tada naučenih polica. Vjerojatnost je veća za najnoviju
naučenu policu i polako se smanjuje s brojem prolaza (množi se s $\beta < 1$).
Za procjenu troška donošenja odluke na stanju $s_t$ (npr.~odluka je određivanje
vrste riječi na trenutnoj riječi) za svaku moguću odluku koristi se
\textit{rollout} polica koja je također stohastička mješavina (linija
\ref{alg:searn:action} i \ref{alg:searn:rolloutpolicy}). Ta polica primjenjuje
se do kraja strukture koju obrađujemo i onda se izračuna dobiveni gubitak
(linija \ref{alg:searn:rollout}). Nakon prošlog postupka za svaku moguću odluku
kreira se primjer za učenje osjetljiv na trošak i stavlja se u skup. Nakon što
je postupak kreiranja primjera gotov nauči se nova polica (linija
\ref{alg:searn:train}). Višerazredni klasifikator koji koristimo mora biti
osjetljiv na trošak, a moguće je koristiti bilo koji binarni klasifikator uz
nekoliko izmjena na primjerima za učenje \citep{zadrozny2003cost,
beygelzimer2005weighted, beygelzimer2005error}. Nakon učenja, zaključivanje se
svodi na korištenje višerazrednog klasifikatora za svako stanje. Ako želimo da
donošenje trenutne odluke ovisi o prethodnim odlukama i značajkama iz budućih
stanja onda je potrebno tijekom učenja i korištenja naučenih polica samo dodati
te odluke i značajke u vektorsku reprezentaciju trenutnog stanja (tijekom
primjene stohastičke police za vrijeme \textit{rollin} i \textit{rollout}
postupka ta reprezentacija mora biti prisutna jer se koristi odabrani
višerazredni klasifikator).

Ovaj algoritam ima vremensku složenost učenja $O(P N T^2 M k)$ -- gdje je $P$
broj prolaza, $N$ broj primjera, $T$ prosječna duljina niza odluka koji
izgrađuju $y \in \mathcal{Y}$, $M$ broj različitih odluka za svako stanje $s_t$
i $k$ broj odluka kojima uvjetujemo trenutnu (svodi se samo na dodavanje $k$
prijašnjih odluka u vektorsku reprezentaciju stanja $s_t$). Vremenska složenost
zaključivanja za jedan primjer je $O(T M k)$. U procjenu učenja nije uključeno
vrijeme učenja klasifikatora generiranim skupom (ako je moguće model učiti
primjer po primjer \engl{online} onda trošak ovisi o broju primjera i njihovoj
vektorskoj reprezentaciji). Kod procjene zaključivanja pretpostavilo se da
klasifikatoru treba $O(M)$ vremena za određivanje točne odluke, a tipična
složenost klasifikatora u tom slučaju je $O(M \cdot d)$ -- gdje je $d$ duljina
vektorske reprezentacije. Ako postoji jednostavna dekompozicija funkcije gubitka
onda složenost učenja može pasti na $O(P N T M k)$. Nije potrebno izvršavati
\textit{rollout} u slučaju da znamo za trenutnu odluku parcijalan gubitak
(npr.~kod označavanja vrste riječi tijekom izvršavanja svih odluka u liniji
\ref{alg:searn:action} ne trebamo izvršiti \textit{rollout} jer znamo da je
gubitak na pojedinoj odluci 1 ako je netočna i 0 ako je točna). Postoji niz
implementacijskih detalja koji mogu, čak i kod nemogućnosti dekompozicije
funkcije gubitka, smanjiti vremensku složenost učenja na $O(P N T M k)$
\citep{daume14lts}.

\textsc{Searn} zahtjeva postojanje optimalne police (potpoglavlje
\ref{ch:policailocalopt}) za potrebe učenja. Ako polica nije optimalna, onda će
učenje biti otežano. U slučaju bolje procjene troška određene odluke potrebno je
izvršiti \textit{rollout} nekoliko puta kako bi povećali vjerojatnost otkrića
točnong niza odluka koji minimizira trošak. Primjer zadataka za kojeg nije lako
izgraditi optimalnu policu je ovisnosno parsiranje i strojno prevođenje. Za
ovisnosno parsiranje nije jednostavno znati kako dovršiti stablo tako da se
minimizira gubitak s obzirom na to da je došlo do niza pogrešaka u izgradnji
parcijalnog stabla. Kod strojnog prevođenja teško je znati kako odrediti
trenutnu riječ s obzirom na polovično preveden dokument. U slučaju da je
dokument preveden krivo možda ima više smisla staviti neku drugu riječ umjesto
one koja je u zlatnom primjeru prisutna na toj poziciji. Kako bi se problem
razriješio u okviru algoritma \textsc{Searn} korišten je Monte-Carlo algoritam
za procjenu troška. Taj postupak usporava učenje, ali omogućava učenje bez
optimalne police. Problem je riješen u \citep{daume15lols} i postupak je opisan
u potpoglavlju \ref{ch:LOLS}. Posljednji nedostatak algoritma je taj što se
moraju čuvati sve naučene police, pristupi opisani u sljedećim potpoglavljima
nemaju to ograničenje.

\citeauthor{daume06thesis} za potrebe svoje disertacije isprobava različite
binarne klasifikatore:
\begin{inlinelist}
  \item perceptron,
  \item logistička regresija,
  \item stroj s potpornim vektorima i linearni kernel i
  \item stroj s potpornim vektorima i kvadratni kernel.
\end{inlinelist}
Izbor klasifikatora ovisi o zadatku. Ako je broj primjera ogroman, onda je bolje
koristiti perceptron, logističku regresiju ili \textsc{svm} s linearnim
kernelom. Ako je broj primjera mali onda bi korištenje \textsc{svm} binarnog
klasifikatora omogućilo što veće margine između odluka i povećalo točnost
(eksperimentalno utvrđeno u disertaciji). Moguće je koristiti i neuronske mreže
s više slojeva, ali trenutno nije u literaturi zabilježeno korištenje. Odabire
binarnih klasifikatora \citeauthor{daume06thesis} uspoređuje sa strukturnim
metodama opisanima u poglavlju \ref{ch:pregled} i pokazuje da \textsc{Searn} ima
istu ili bolju učinkovitost na odabranim zadacima.

\citeauthor{daume06thesis} tvrdi da, u kontekstu algoritama za strukturno
predviđanje, \textsc{Searn} leži između algoritama koji koriste globalnu
normalizaciju, kao što su \mmmm{} i \textsc{crf}, i onih koji koriste lokalnu
normalizaciju, opisani u \citep{punyakanok2001use}. Razlika između
\textsc{Searn} i globalnih algoritama je u načinu na koji rješavaju nesigurnost.
Kod globalnih algoritama, algoritam pretraživanja se koristi za vrijeme
zaključivanja kako bi se razriješila nesigurnost. Kod algoritma \textsc{Searn},
trošak za svaku odluku koji je prisutan tijekom učenja razrješava nesigurnost.
Oba pristupa se razlikuju od lokalnih algoritama kod kojih se nesigurnost uopće
ne razrješava.

Sa šire perspektive strojnog učenja, \textsc{Searn} je pokazao povezanost između
podržanog učenja i strukturnog predviđanja. Strukturno predviđanje može se
vidjeti kao problem podržanog učenja u kojem su sva opažanja (sve prave odluke)
vidljive na početku.

Sve navedeno u zadnja dva paragrafa vrijedi i za ostale metode učenja
pretraživanja navedene u ovom poglavlju. \citeauthor{daume06thesis} postavlja
otvorena pitanja vezana uz broj potrebnih primjera koji je potreban za učenje i
što raditi bez prisustva optimalne police. Odgovori na ta pitanja dani su
algoritmom \textsc{LOLS} koji je opisan u poglavlju \ref{ch:LOLS}
\citep{daume15lols}.

\textsc{Searn} se može opisati, koristeći terminologiju uvedenu u potpoglavljima
\ref{ch:reductions}, \ref{ch:policailocalopt} i \ref{ch:rollinrollout}, kao
konzistentna (ako je referentna polica optimalna) redukcija binarne
klasifikacije na problem združenog predviđanja gdje se za \textit{rollin} i
\textit{rollout} policu koristi stohastička mješavina između naučene i
referentne police.
