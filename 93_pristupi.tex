U nastavku slijedi opis metoda učenja pretraživanja. Pristupi opisani u ovom
poglavlju zahtijevaju sljedeće.

\begin{itemize}

  \item \textbf{Dobru referentnu policu.} Svi pristupi -- osim \textsc{lols} --
  zahtijevaju optimalnu referentnu policu inače može doći do nakupljanja
  pogreške \engl{compounding error};

  \item \textbf{Dobar binarni klasifikator.} Za učenje pretraživanja potrebno je
  efikasno rješenje za problem višerazredne klasifikacije osjetljive na trošak.
  Koristeći težinsku \textsc{ovo} ili \textsc{ova} \engl{weighted all-pairs or
  one-against-all} redukciju iz \citep{beygelzimer2005weighted,
  beygelzimer2005error} u kombinaciji s \textit{costing} algoritmom iz
  \citep{zadrozny2003cost} moguće je koristiti bilo koji binarni klasifikator
  bez korigiranja izvornog koda klasifikatora. Moguće je koristiti i pravi
  višerazredni klasifikator osjetljiv na trošak bez navedenih tehnika redukcije;

  \item \textbf{Dobro definiranu funkciju gubitka.} Ona može biti definirana na
  cijeloj strukturi koju predviđamo. Na primjer, može se između predviđenog i
  zlatnog stabla računati \textunderscript{F}{1} gubitak. Između prevedenog i
  referentnog prijevoda može se koristiti BLEU \engla{bilingual evaluation
  understudy}{BLEU} mjeru. Prethodne navedene funkcije gubitka nemaju
  dekompoziciju preko strukture odluka. Moguće je koristiti i funkcije poput
  Hammingovog gubitka.

\end{itemize}
