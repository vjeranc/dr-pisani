\documentclass[times, utf8, diplomski]{fer}
\usepackage{booktabs}
%% for clickable pdf with lovely unicode letters
\usepackage[unicode]{hyperref}

%% drawing various dependency trees and graphs
\usepackage{tikz-dependency}
\usepackage{tkz-graph}

%% rollin-rollout graphs
\usepackage{tikz}
\usetikzlibrary{arrows.meta,decorations.pathreplacing,matrix,positioning}
\colorlet{lightblue}{blue!60!white}
\colorlet{darkgreen}{green!60!black}

%% subfigure and subcaptions
\usepackage{subcaption}
%% enumeration inline
\newlist{inlinelist}{enumerate*}{1}
\setlist*[inlinelist,1]{%
  label=(\arabic*),
}

%% times math
\usepackage{newtxtext}
\usepackage{newtxmath}

\usepackage{multirow} %% tablica
\usepackage{colortbl}
\usepackage{xcolor}
\usepackage{textcomp}
%% bolds the math consistently (italic, functions)
\newcommand{\mbold}[1]{{\boldmath #1}}

%% l2s
\def \lts {\textsc{l2s}}

%% for notes
\newcommand{\bq}[2]{{\fbox{#1}\colorbox{red}{~#2}}}

%% engl
\newcommand{\engla}[2]{(engl.~\emph{#1}, \textsc{#2})}

%% for exponents to differantiate outputs inputs and hypotheses
%% x\ssymbol{1} called like this
\def\@fnsymbol#1{\ensuremath{\ifcase#1\or *\or \dagger\or \ddagger\or
   \mathsection\or \mathparagraph\or \|\or **\or \dagger\dagger
   \or \ddagger\ddagger \else\@ctrerr\fi}}

\newcommand{\ssymbol}[1]{^{\@fnsymbol{#1}}}


%% argmin argmax operator
\DeclareMathOperator*{\argmax}{argmax}
\DeclareMathOperator*{\argmin}{argmin}

%% condition
\newtheorem{condition}{Uvjet}
%% definition
\newtheorem{definition}{Definicija}
%% definition
\newtheorem{theorem}{Teorem}

%% fields
\newcommand*{\field}[1]{\mathbb{#1}}
%% max margin markov networks abbrev
\newcommand*{\mmmm}[0]{$\text{M}^4$}
%% text underscript
\newcommand{\textunderscript}[2]{$\text{#1}_{\text{#2}}$}

%% url font style
\urlstyle{sf}

%% code listings
\usepackage{listings}

%% algorithm

\usepackage{algorithm}
\usepackage{algpseudocode}

% algorithm counter resets every chapter
\makeatletter
\@addtoreset{algorithm}{chapter}
\makeatother
\renewcommand{\thealgorithm}{\thechapter.\arabic{algorithm}}
% Algorithm # is <chapter>.<algorithm>

\algrenewcommand{\algorithmicrequire}{\textbf{Potrebno:}}
\algrenewcommand{\algorithmicend}{\textbf{kraj}}
\algrenewcommand{\algorithmicif}{\textbf{ako}}
\algrenewcommand{\algorithmicthen}{\textbf{onda}}
\algrenewcommand{\algorithmicelse}{\textbf{inače}}
\algrenewcommand{\algorithmicfor}{\textbf{za}}
\algrenewcommand{\algorithmicforall}{\textbf{za svaki}}
\algrenewcommand{\algorithmicforall}{\textbf{za svaki}}
\algrenewcommand{\algorithmicfunction}{\textbf{funkcija}}
\algrenewcommand{\algorithmicdo}{}
\algrenewcommand{\algorithmicwhile}{\textbf{dok je}}
\algrenewtext{EndFor}{\textbf{kraj petlje}}
\algrenewtext{EndWhile}{\textbf{kraj petlje}}
\algrenewtext{EndIf}{\textbf{kraj}}
\algrenewtext{EndFunction}{\textbf{kraj funkcije}}
\algrenewcommand{\algorithmicreturn}{\State \textbf{vrati}}

\algdef{SE}[WHILE]{While}{EndWhile}[1]
  {\algorithmicwhile\ #1\ \textbf{ponavljaj}}
  {\textbf{kraj petlje}}

\makeatletter
\renewcommand{\ALG@name}{Algoritam}
\renewcommand{\listalgorithmname}{Popis algoritama}
\makeatother

\errorcontextlines\maxdimen

% http://tex.stackexchange.com/questions/144840/vertical-loop-block-lines-in-algorithmicx-with-noend-option
% begin vertical rule patch for algorithmicx (http://tex.stackexchange.com/questions/144840/vertical-loop-block-lines-in-algorithmicx-with-noend-option)
\makeatletter
% start with some helper code
% This is the vertical rule that is inserted
\newcommand*{\algrule}[1][\algorithmicindent]{\makebox[#1][l]{\hspace*{.5em}\vrule height .75\baselineskip depth .25\baselineskip}}%

\newcount\ALG@printindent@tempcnta
\def\ALG@printindent{%
    \ifnum \theALG@nested>0% is there anything to print
        \ifx\ALG@text\ALG@x@notext% is this an end group without any text?
            % do nothing
            \addvspace{-3pt}% FUDGE for cases where no text is shown, to make the rules line up
        \else
            \unskip
            % draw a rule for each indent level
            \ALG@printindent@tempcnta=1
            \loop
                \algrule[\csname ALG@ind@\the\ALG@printindent@tempcnta\endcsname]%
                \advance \ALG@printindent@tempcnta 1
            \ifnum \ALG@printindent@tempcnta<\numexpr\theALG@nested+1\relax% can't do <=, so add one to RHS and use < instead
            \repeat
        \fi
    \fi
    }%
\usepackage{etoolbox}
% the following line injects our new indent handling code in place of the default spacing
\patchcmd{\ALG@doentity}{\noindent\hskip\ALG@tlm}{\ALG@printindent}{}{\errmessage{failed to patch}}
\makeatother
% end vertical rule patch for algorithmicx

\usepackage{pdfpages}

\begin{document}

% TODO: Navedite broj rada.
\thesisnumber{1323}

% TODO: Navedite naslov rada.
\title{Učenje pretraživanja za rješavanje zadataka obrade prirodnog jezika}

% TODO: Navedite vaše ime i prezime.
\author{Vjeran Crnjak}

\maketitle

% Ispis stranice s napomenom o umetanju izvornika rada. Uklonite naredbu \izvornik ako želite izbaciti tu stranicu.
\izvornik

% Dodavanje zahvale ili prazne stranice. Ako ne želite dodati zahvalu, naredbu ostavite radi prazne stranice.
\zahvala{}

\tableofcontents
\listoffigures
\listoftables

Metode učenja pretraživanja \engl{learning to search, abbrev.~\textsc{L2S}}
novija su pojava u strojnom učenju, ali u zadnjih deset godina istraživači su
metode primijenili na probleme u različitim granama računarske znanosti,
uključujući i obradu prirodnog jezika. Vrlo uspješna je primjena vezana uz
probleme gdje se učenje i zaključivanje odvija na primjerima koji sadrže
određenu strukturu. Učenje pretraživanja spada u metode strukturnog učenja
\engl{structured learning} ili predviđanja strukturiranog izlaza
\engl{structured output prediction}. U obradi prirodnog jezika jedni od
glavnih problema koji uključuju predviđanje strukturiranog izlaza su označavanje
vrste riječi \engl{part-of-speech tagging}, ovisnosno parsanje
\engl{dependency parsing}, prepoznavanje imenovanih entiteta \engl{named
entity recognition}, prepoznavanje i praćenje entiteta \engl{entity tracking
and detection}, razrješavanje koreference \engl{coreference resolution},
izlučivanje relacija između entiteta \engl{entity relation extraction},
inkrementalna prevođenje \engl{incremental translation}, inkrementalno
odgovaranje na pitanja \engl{incremental question answering} i mnoštvo
drugih. Ostale primjene metoda učenja pretraživanja su aktivno i interaktivno
učenje \engl{active and interactive learning}, algoritmi grananja i granice
\engl{branch-and-bound}, robotika, segmentacija slike \engl{image
segmentation}, predviđanje sekundarne strukture proteina \engl{protein
secondary structure prediction} itd.

Jedan razlog mnoštvu primjena u tako kratkom vremenskom roku je matematička
priroda metoda. Metode učenja pretraživanja vuku inspiraciju iz područja
podržanog učenja \engl{reinforcement learning} gdje je zadatak sustava
naučiti dobru \bq{policu}{Dobar prijevod -- \emph{policy}?}, a u slučaju učenja
pretraživanja cilj je iskoristiti prisutne podatke, dok su ti podaci zanemareni
ili se postepeno izgrađuju kod primjene podržanog učenja. Drugi razlog je
primjena matematičkog aparata koji je nastao da bi objasnio podržano učenje i
moć redukcija. Redukcija u području strojnog učenja je slična redukcijama u
drugim područjima računarske znanosti. Redukcija 3-SAT problema na problem
pronalaženja Hamiltonovog ciklusa, gdje je prethodni također redukcijom dokazano
NP-potpun, omogućuje zaključak da je i problem Hamiltonovog ciklusa NP-potpun
(precizno je reći da dokaz NP-potpunosti mora uključivati i algoritam
polinomijalne složenosti za provjeru rješenja). U kontekstu strojnog učenja
redukcije se razlikuju. Pitanje je možemo li neki teži problem kao što je
višerazredna klasifikacija izvesti s dobrim garancijama koristeći binarni
klasifikator. Također, pitanje je kako reducirati združeno učenje i predviđanje
\engl{joint learning and prediction} na jednostavniji problem. Naposljetku,
glavni razlog široke primjene je vrlo jednostavna implementacija koja
iskorištava svojstvo modularnosti koje je prirodno u procesu redukcije (za
implementaciju višerazredne klasifikacije potreban nam je samo dobar binarni
klasifikator). Implementacija je ostvarena u brzom sustavu za strojno učenje
zvanom Vowpal Wabbit, a \textsc{L2S} razvojni okvir omogućava da u jako malo
linija programskog koda istraživač može napisati algoritam učenja i
zaključivanja za specifični problem. Slični sustavi koji implementiraju
algoritme učenja i zaključivanja za vjerojatnosne grafičke modele zahtijevaju
puno više linija programskog koda i vjerojatnija je pojava bugova, a nemodularna
priroda onemogućuje njihovo izbjegavanje. Pokušaj generalizacije i
iskorištavanja programskog koda općenitijim metodama učenja i zaključivanja kod
vjerojatnosnih grafičkih modela rezultiralo je pojavom vjerojatnosnih
programskih jezika \engl{probabilistic programming languages}, ali brzina
učenja i zaključivanja nije ni blizu algoritama \textsc{L2S}.

Ovaj diplomski rad nudi sažet pregled metoda učenja pretraživanja i njihovu
primjenu na probleme obrade prirodnog jezika. Uz njihovu matematičku podlogu
obrađene su i usporedbe s ostalim metodama strojnog učenja koje se uspješno
primjenjuju na istim problemima. Dani su i odgovori na prijašnja postavljena
pitanja te opisi konkretne implementacije koji pažljivije argumentiraju
prednosti nad ostalim pristupima.

U \bq{$\ldots$}{tom poglavlju nađemo ovo i ono, blabla.}


\chapter{Pregled pristupa združenom učenju}

\section{Združeno učenje}

\section{Vjerojatnosni grafički modeli}

\section{Markovljev model maksimalne entropije}

\section{Strukturirani perceptron}

\section{Markovljeve mreže maksimalne margine}

\section{Structured SVM}

\chapter{Učenje pretraživanja}

\section{Učenje redukcija}
\section{SEARN}
\section{LOLS}
\section{DAgger}
\section{Aggravate}

\chapter{Primjena na konkretan zadatak}

\section{Označavanje vrste riječi}

\section{Parsiranje ovisnosnog stabla}

\section{Označavanje imenovanih entiteta}

\chapter{Vrednovanje modela za hrvatski jezik}

\section{Označavanje vrste riječi}

\section{Označavanje morfosintaktičkih opisnika}

\subsection{Označavanje po razinama}

\subsection{Označavanje s više prolaza}

\section{Parsiranje ovisnosnih stabala}

\subsection{Dvorazinsko parsiranje}

\subsection{Združeno parsiranje}

\chapter{Zaključak}

Ovdje zaključujem.


\bibliography{literatura}
\bibliographystyle{fer}


\begin{sazetak}
Sažetak na hrvatskom jeziku.

\kljucnerijeci{Ključne riječi, odvojene zarezima.}
\end{sazetak}


\engtitle{Learning to search for natural language processing problems}
\begin{abstract}
Abstract.

\keywords{learning to search, natural language processing}
\end{abstract}


\nocite{daume06thesis}
\nocite{daume09searn}
\nocite{daume06searn-practice}
\nocite{daume15reductions}
\nocite{daume15lols}
\nocite{daume15rewrite}
\nocite{daume14lts}

\end{document}
