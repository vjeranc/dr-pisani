Politika \engl{policy} termin je za naučen model nakon postupka učenja ili
referentni sustav ili algoritam koji vrši određeno ponašanje. U području
podržanog učenja jedina politika koja postoji je ona koja je naučena. U području
učenja oponašanjem \engl{imitation learning} postoji više vrsta politika. Politika
koja je naučena, referentna politika koju se pokušava oponašati i prava optimalna
politika (stručnjak). U nastavku slijedi nekoliko definicija koje bi trebale
raščistiti koncept politika.

\begin{definition}[Politika]

  Politika $\pi$ je distribucija preko odluka koje su uvjetovane ulazom $x$ i
  stanjem $s$.

\end{definition}

\noindent
Skoro svi pristupi u potpoglavlju \ref{ch:metodesearchlearn} zahtijevaju pristup
efikasnoj i optimalnoj politici $\pi\ssymbol{1}$. Bez takve politike učenje može
biti dugotrajno ili redukcija nekonzistentna. Potrebno je definirati optimalnu
politiku.

\begin{definition}[Optimalna politika]\label{def:optimalpolicy}

  Za par $(x, \mathbf{c})$ iz definicije \ref{def:jointlearn} i stanje $s =
  \langle y_1, \ldots, y_t \rangle$ u prostoru pretraživanja optimalna politika
  $\pi\ssymbol{1}(x, \mathbf{c}, s)$ je $\argmin_{y_{t+1}} \min_{y_{t+2},
  \ldots, y_T} c_{\langle y_1, \ldots, y_T \rangle}$. Tj.~$\pi\ssymbol{1}$
  odabire odluku (vrijednost za $y_{t+1}$) koja minimizira cjelokupni gubitak
  pretpostavljajući da će sve ostale buduće odluke biti donesene optimalno.

\end{definition}

\noindent
Lokalno optimalno učenje pretraživanja -- \textsc{LOLS} (potpoglavlje
\ref{ch:LOLS}) --  ima garancije o tome kakva će naučena politika biti, a
definicija je dana ispod.

\begin{definition}[Lokalno optimalna politika]

  Naučena politika je lokalno optimalna ako promjena bilo koje prijašnje odluke
  ne može poboljšati učinkovitost naučene politike.

\end{definition}

\noindent
Potrebno je znati težinu učenja lokalno optimalne politike. Možda je problem
lokalne optimalnosti toliko težak da bilo koji efikasan algoritam učenja neće
moći naučiti lokalno optimalnu politiku u zadovoljavajućoj količini vremena.
Teorem koji slijedi govori o tome koliko je ažuriranja potrebno napraviti da bi
naučena politika bila lokalno optimalna.

\begin{theorem} \label{th:localopt}

  Pretpostavimo dostupnost algoritma koji ažurira politike koristeći odstupanje
  od jednog koraka od trenutne politike. Onda postoji problem s prostorom
  pretraživanja, razred politike i funkcija gubitka gdje takav algoritam mora
  napraviti $\Omega(2^T)$ ažuriranja prije nego što nauči lokalno optimalnu
  politiku.

\end{theorem}

Konstrukcija problema koja dokazuje teorem \ref{th:localopt} prisutna je u
\citep{daume15lols}. Naravno, kod problema gdje je moguće napraviti
dekompoziciju prostora odluka i funkcije gubitka -- kao što je slučaj kod
označavanja vrste riječi -- ne postoji konstrukcija koja bi zahtijevala
eksponencijalno mnogo primjera. Ilustrativniji je primjer igranja šaha. Mnoštvo
metoda koristi, uz bazu odigranih partija i dobro analiziranih otvaranja,
pretraživanje tijekom igre za pronalazak najboljih poteza. Pitanje je može li se
naučiti šah bez potrebe za pretraživanjem tijekom igre? Može li se naučiti
politika koja će biti lokalno optimalna? Takva politika bi garantirala da nakon
odigrane partije šaha ne postoji potez koji se može ispraviti i time promijeniti
rezultat igre. Tj.~ako bi postojao takav potez i ako bi se ažurirala naučena
politika, onda ona ne bi mogla postati bolja. U slučaju šaha očito je da bi politika
trebala postati bolja. Ako politika za svako stanje ploče može reći optimalnu
vjerojatnost dobitka partije, onda bi otkriće boljeg poteza tu vjerojatnost
trebalo promijeniti. Očigledno je da naučena politika za šah može biti lokalno
optimalna jedino u slučaju da je odigran ogroman broj partija. Moguće je da šah
kojeg igraju ljudi nije šah kojeg bi igrala sveznajuća računala. Potezi koje
rade ljudi imaju određenu distribuciju gdje bi učenje lokalno optimalne politike
koja igra protiv čovjeka zahtijevalo manje primjera. Dobar stvaran primjer je
nedavna pobjeda AlphaGo protiv ljudskog igrača u igri Go.
\citet{silver2016mastering} izgrađuju politiku koja za svako stanje vraća
vjerojatnost pobjede uzevši u obzir sljedeći potez. S obzirom na broj mogućih
igara ($\approx 2\cdot10^{170}$), vrlo je teško izgraditi takvu politiku čija će
procjena vjerojatnosti biti lokalno optimalna te za poboljšanje učinkovitosti
ipak koriste pretragu tijekom zaključivanja.
