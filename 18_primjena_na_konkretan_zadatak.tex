U poglavlju opisane su primjene metoda učenja pretraživanja na zadatke u obradi
prirodnog jezika. Razmatrane probleme moguće je riješiti na više načina gdje
svaki ima svoje prednosti i mane. Dana je usporedba s metodama koje su korištene
za rješavanje tih problema koja demonstrira superiornost metoda učenja
pretraživanja. Metode učenja pretraživanja omogućuju da istraživač u području
obrade prirodnog jezika ne treba biti istovremeno i istraživač u području
strojnog učenja (ne treba znati kako efikasno implementirati algoritme učenja i
zaključivanja kod raznih modela). Priloženi pseudokodovi koji slijede trebali bi
potvrditi izjavu u prošloj rečenici.
