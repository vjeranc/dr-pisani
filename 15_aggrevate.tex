\citet{ross2011reduction} predstavljaju metodu učenja pretraživanja
\textsc{AggreVaTe} \engla{aggregate values to imitate}{AggreVaTe} koja je
proširenje algoritma \textsc{DAgger}. Proširenje se svodi na to da uz referentnu
policu koja se koristi za odabir sljedećih poteza, tijekom učenja je moguće
koristiti i stručnjaka koji želi što manje biti u interakciji s algoritmom
učenja. Tj.~odabir da odluke u kojoj bi stručnjak trebao vrednovati preostali
niz odluka ima određeni trošak. Cilj je istovremeno naučiti dobru policu, ali
pri učenju što manje koristiti stručnjaka. U stanju u kojem je skupo procijeniti
gubitak odluke (ili informacija o trošku ne postoji) može se pozvati stručnjak
koji će vratiti točnu odluku i njen trošak.

Dobar primjer koji opisuje primjenu ove metode je povećanje klikovnog postotka
\engla{click-through rate}{ctr}. Stručnjak je u ovom slučaju dizajner koji
pokušava prilagoditi web-stranicu posjetitelju, a cilj je povećati broj klikova
na oglase koje korisnik vidi. Postupak kojim bi se potencijalno povećali klikovi
je mijenjanje izgleda web-stranice (pozicioniranje oglasa, veličina oglasa,
sadržaj uz koji se oglas pojavljuje i dr.). Želi se izbjeći mogućnost da
korisnik zbog ogromnog broja nasumičnih promjena odluči ne posjećivati stranicu.
U tom je slučaju potrebno vršiti eksperiment promjene stranice što manje puta i
pratiti kako promjene utječu na klikovni postotak (nakon nekog vremena od
trenutka promjene se izračuna količina promjene \textsc{ctr}).

Drugi primjer je interaktivno označavanje skupa podataka. Pretpostavimo
prisutnost malog označenog skupa podataka. Cilj je naučiti model koji će imati
dobru generalizaciju na tom skupu. Za algoritam \textsc{AggreVaTe} odaberemo
veći neoznačeni skup. Cilj algoritma istovremeno minimizirati broj interakcija
sa stručnjakom i gubitak na malom označenom skupu podataka (postojao bi i treći
skup na kojem bi provjerili stvarnu mogućnost generalizacije).

Algoritam ima nedostatak u kojem je moguće da stručnjak daje preoptimistične
procjene nakon izvršene odluke. Autori ilustriraju to primjerom gdje se uči
vožnja automobila. Stručnjak bi vjerojatno odabrao užu, ali kraću cestu, za
dolazak do cilja, unatoč postojanju duže, ali sigurnija ceste koja vodi do istog
cilja. Ako naučena polica na užoj cesti dođe u stanje za koju ne radi dobro
predviđanje troška sljedeće odluke, onda je moguće da će vozilo otići niz
liticu. Navedeni problem može se riješiti tako da se poveća istraživanje
dolazaka do svih stanja (problem je ekvivalentan pristranosti oznakama).

Ovaj algoritam je interaktivna verzija algoritma \textsc{DAgger} stoga je za
učenje iz podataka dovoljan potonji. \textsc{AggreVaTe} je opisan radi usporedbe
s algoritmom \textsc{LOLS}.
