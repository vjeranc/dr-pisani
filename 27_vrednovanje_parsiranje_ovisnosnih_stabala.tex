Za vrednovanje učinkovitosti koristi se mjera točnosti \engl{accuracy}. Stablo
ne mora imati oznake na bridovima te se onda koristi mjera bez oznaka
\engla{unlabeled attachment score}{uas}. Ako su usmjereni bridovi označeni, onda
se koristi mjera točnost uzimajući oznake u obzir \engla{labeled attachment
score}{las}. usporedbu se koriste rezultati dobiveni u \citep{agic2013three}
koji su dobiveni MSTParser označivačem. Vrednovanje se vrši na testnom skupu u
CoNLL-U formatu.\footnote{Datoteka \texttt{hr-ud-test.conllu} prisutna na
poveznici \url{https://github.com/UniversalDependencies/UD_Croatian}} Rezultati
nisu usporedivi jer se koristi drugi podatkovni skup za MSTParser rezultate. Taj
podatkovni skup ima samo 15 oznaka dok korišteni skup ima 32. Zbog toga bi
problem označavanja UD skupa trebao biti teži. Oba skupa imaju iste rečenice i
ista ovisnosna stabla bez oznaka, a testni skup prikazan u tablicama je spojeni
\textsc{SETimes} i \textsc{Wiki}.

Korištene značajke uključuju sufikse i prefikse do najveće duljine od pet
znakova. To su jedine značajke koje je moguće kontrolirati izvan algoritma.
Model nakon treniranja koristi više od milijardu zasebnih značajki (svaki
jedinstveni sufiks i prefiks i interne značajke su u tom broju) dok je npr.~za
označavanje vrste riječi taj broj oko 40 milijuna. Za pregled ostalih internih
značajki čitatelja se upućuje na \citep{chang2015learning}. Kako je model
prevelik koristi se neuronska mreža s jednim skrivenim slojem od pet čvorova, a
za regularizaciju koristi se postupak prati-regulariziranog-vođu \engla{follow
the regularized leader}{ftrl} koji uključuje L1 i L2 regularizaciju. Prikazani
rezultati koriste navedenu konfiguraciju inače su rezultati lošiji jer je skup
za učenje premali s obzirom na broj značajki. \citet{chang2015learning} ne
koriste sufikse i prefikse jer rade označavanje na jezicima koji imaju dovoljno
podataka -- značajke su samo cijele riječi i oznaka vrste riječi.

U tablici \ref{table:depparsing} prikazani su rezultati. Kako morfosintaktički
deskriptori sadrže informaciju o sintaksi očekivano je da njihovo korištenje
poboljšava učinkovitost oba pristupa. Razlog zašto je \textsc{Vwdep} bolji je
zbog superiornijeg algoritma. \citet{cer2010parsing} dobivaju slične razlike
između pristupa baziranog na minimalnom razapinjućem stablu i tranzicijskog
parsera. Prikazani rezultati su optimistični jer pretpostavljaju točne oznake
vrste riječi kao ulaz, ali označivač vrste riječi razvijen u okviru ovog rada
dovoljno je točan da nema bitne razlike između parsera koji koristi zlatne
oznake.

\begin{table}
\centering
\caption{Rezultat ovisnosnog parsiranja.}
\label{table:depparsing}
\begin{tabular}{|l|c|c|}
\hline
Metoda               & \textsc{las}   & \textsc{uas}    \\ \hline \hline
MSTParser      (POS) & 74.56          & 81.59           \\
MSTParser      (MSD) & 77.49          & 83.58           \\
\textsc{Vwdep} (POS) & 74.91          & 83.17           \\
\textsc{Vwdep} (MSD) & \textbf{79.22} & \textbf{86.44}  \\ \hline
\end{tabular}
\end{table}
