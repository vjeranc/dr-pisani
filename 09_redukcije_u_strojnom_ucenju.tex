Redukcije u strojnom učenju postupak su za transformiranje jednostavnijih
problema u složenije -- kao što je primjer redukcije problema binarne
klasifikacije u problem višerazredne klasifikacije gdje se može koristiti
jedan-protiv-svih \engla{one-against-all}{ova} ili jedan-protiv-drugog
\engla{one-against-one}{ovo} pristup korištenja binarnih klasifikatora. Za
\textsc{ova} pristup potrebno je $n$ klasifikatora gdje je $n$ broj razreda, a
\textsc{ovo} zahtijeva $\binom{n}{2}$ klasifikatora. Prethodni ima prednost u
vremenu ako je broj razreda velik, a potonji ako je vektor gust za \textsc{ova},
a rijedak za \textsc{ovo} (ako je moguće izbaciti značajke tj.~zadržati samo one
koje su vezane za dva razreda koja uspoređujemo). Ostale prednosti ovise o samom
problemu, a čak i o vrsti binarnog klasifikatora kojeg koristimo
\citep{milgram2006one}. Zanimljiv je primjer redukcija problema višerazredne
klasifikacije stablom odluka \engl{decision tree} gdje se svaki razred može
binarno kodirati, a bitovi predstavljaju niz da/ne odgovora o tome kojim bridom
krenuti gdje bi se nakon $O(\log n)$ koraka potpuno kodirao razred
\citep{beygelzimer2009error, daume2016one}. Ovo je eksponencijalno ubrzanje s
obzirom na prethodne redukcije u vremenu, a i broj klasifikatora je puno manji -- $\lceil \log n \rceil$.

Pitanje je jesu li redukcije učinkovit pristup za rješavanje složenih problema
strojnog učenja? Odgovor nije očit jer je teško opisati reprezentaciju na
trivijalan matematički način. Moguće je da redukcija stvori jednostavne probleme
koje nije lako riješiti. Trivijalan je primjer trorazredne klasifikacije s
jednom značajkom i linearnim binarnim klasifikatorom. Kod \textsc{ova} pristupa
nije moguće odvojiti razrede s dovoljno malom pogreškom, a kod \textsc{ovo} je
to moguće.

Redukcije nisu toliko trivijalne kao ovi primjeri. Postoje tri potrebne
komponente:
\begin{inlinelist}
  \item preslikavanje hipoteze,
  \item preslikavanje primjera i
  \item analiza ograde.
\end{inlinelist}
Preslikavanje hipoteze opisuje postupak kako pretvoriti rješenje za jednostavan
problem u rješenje za složen. Preslikavanje primjera opisuje postupak kako
stvoriti skupove podataka za jednostavan problem koristeći skupove podataka za
složen problem. Ograda daje jamstvo učinkovitosti na složenom problemu ako postoji
dobra učinkovitost na jednostavnom problemu.

Postoje dvije vrste ograda koje se uzimaju u obzir: ograde na ograničenje
pogreške \engl{error bounds} i ograde na ograničenje žaljenja \engl{regret
bounds}. U slučaju redukcije koja ograničava pogrešku, teoretsko jamstvo tvrdi
da mala pogreška na jednostavnom problemu podrazumijeva malu pogrešku na složenom
problemu. U slučaju redukcije koja ograničava žaljenje, ograda tvrdi da malo
žaljenje na jednostavnom problemu podrazumijeva malo žaljenje na
složenom.\footnote{Žaljenje hipoteze $h$ na problemu $\mathcal{D}$ je razlika u
pogrešci kod korištenja $h$ i korištenja najboljeg mogućeg klasifikatora.
Formalno, $R(\mathcal{D}, h) = L(\mathcal{D}, h) - \min_{h\ssymbol{1}}
L(\mathcal{D}, h\ssymbol{1})$.} Zadovoljavajuća je činjenica da ograde imaju
svojstvo kompozicije \citep{beygelzimer2005error}. Ako je moguće reducirati
problem $X$ u problem $Y$ s ogradom $f$ i problem $Y$ u problem $Z$ s ogradom
$g$, onda je kompozicija $X \circ Y$ redukcija s ogradom $f \circ g$. Navedeno
svojstvo pojednostavljuje matematičku analizu ograda redukcije. Bitna je razlika
između navedenih ograda što ako redukcija s ograničenjem pogreške kreira
jednostavne probleme koji su šumoviti, onda su bilo kakve tvrdnje o težem
problemu isprazne.

Potrebno je znati je li redukcija konzistentna \engl{consistent} da bi se
uvjerili da će redukcija raditi dobro \citep{beygelzimer2009error,
daume15reductions}. Neformalno, redukcija koja transformira bilo koje optimalno
rješenje za osnovni jednostavni problem u optimalno rješenje za složen
problem je \textit{konzistentna}. Ispostavlja se da su gore navedene jednostavne
redukcije stablom odluke i \textsc{ova} pristup nekonzistentne. Navedene uvjete
nekad je moguće zadovoljiti ne samo odabirom načina zaključivanja (\textsc{ovo},
\textsc{ova} i dr.) nego i načinom učenja \citep{abe2004iterative,
beygelzimer2005weighted} -- način učenja je glavni razlog mogućnosti redukcije
višerazredne klasifikacije u strukturno predviđanje. Konzistentnost je osnovni
uvjet za dobru redukciju, a redukcije s ograničenjem pogreške su, na žalost,
uglavnom nekonzistentne. Čitatelja se upućuje na sustavni pregled zadnjih deset
godina redukcija u strojnom učenju \citep{daume15reductions}.

Redukcije su bitna sastavnica u najuspješnijim \lts{} metodama jer bez njih ne
bi postojala informacija o tome zašto su toliko uspješne, nego bi
pretpostavljali da koristimo heurističke metode. Heurističke metode prisutne su
u izobilju kod primjena na probleme združenog predviđanja i bez teorije postoji
samo empirijska potvrda njihove uspješnosti.\footnote{Dobar primjer je razvoj
metoda za ovisnosno parsiranje koje koriste \textit{beam} pretragu i time
eksperimentalno poboljšavaju performanse \citep{zhang2011transition,
bohnet2012transition}.}
