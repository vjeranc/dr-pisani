%% for clickable pdf with lovely unicode letters
\usepackage[unicode]{hyperref}

%% drawing various dependency trees and graphs
\usepackage{tikz-dependency}
\usepackage{tkz-graph}

%% rollin-rollout graphs
\usepackage{tikz}
\usetikzlibrary{arrows.meta,decorations.pathreplacing,matrix,positioning}
\colorlet{lightblue}{blue!60!white}
\colorlet{darkgreen}{green!60!black}

%% times math
\usepackage{newtxmath}

\usepackage{multirow} %% tablica
\usepackage{colortbl}
\usepackage{xcolor}
\usepackage{textcomp}
%% bolds the math consistently (italic, functions)
\newcommand{\mbold}[1]{{\boldmath #1}}

%% l2s
\def \lts {\textsc{l2s}}

%% for notes
\newcommand{\bq}[2]{{\fbox{#1}\colorbox{red}{~#2}}}

%% for exponents to differantiate outputs inputs and hypotheses
%% x\ssymbol{1} called like this
\def\@fnsymbol#1{\ensuremath{\ifcase#1\or *\or \dagger\or \ddagger\or
   \mathsection\or \mathparagraph\or \|\or **\or \dagger\dagger
   \or \ddagger\ddagger \else\@ctrerr\fi}}

\newcommand{\ssymbol}[1]{^{\@fnsymbol{#1}}}


%% argmin argmax operator
\DeclareMathOperator*{\argmax}{argmax}
\DeclareMathOperator*{\argmin}{argmin}

%% condition
\newtheorem{condition}{Uvjet}
%% definition
\newtheorem{definition}{Definicija}

%% fields
\newcommand*{\field}[1]{\mathbb{#1}}

%% text underscript
\newcommand{\textunderscript}[2]{$\text{#1}_{\text{#2}}$}

%% url font style
\urlstyle{sf}

%% algorithm

\usepackage{algorithm}
\usepackage{algpseudocode}

\makeatletter
\renewcommand{\ALG@name}{Algoritam}
\renewcommand{\listalgorithmname}{Popis \ALG@name a}
\makeatother
