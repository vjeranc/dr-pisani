U ovom poglavlju definiran je pojam strukturnog i združenog predviđanja i
učenja. Dan je kratak pregled svih popularnih metoda koje su se primjenjivale na
probleme strukturnog predviđanja u području obrade prirodnoga jezika:
\begin{enumerate}
  \item vjerojatnosni grafički modeli \engl{probabilistic graphical models}},
    \item skriveni Markovljev model \engla{hidden Markov model}{hmm},
    \item Markovljev model maksimalne entropije \engla{maximum entropy Markov
    model}{memm},
    \item uvjetna slučajna polja \engla{conditional random fields}{crf},
  \item strukturirani perceptron \engl{structured perceptron},
  \item Markovljeve mreže maksimalne margine \engla{maximum margin Markov
  networks}{\mmmm{}} i
  \item strukturirani stroj potpornih vektora.
\end{enumerate}
Napravljena je i detaljnija analiza sposobnosti modela koji slijede i detaljan
opis mana koje objašnjavaju zašto metode nisu dovoljno generalne da bi bile
dobre za široku primjenu na problemu strukturnog predviđanja. Poglavlje koje
slijedi definira i opisuje problem združenog predviđanja na koji se gornji
modeli primijenjuju. Sustavni pregled temeljen je na disertaciji
\citep{daume06thesis}.
