\citet*{daume15lols} uvode \textit{rollin} i \textit{rollout} terminologiju u
kontekst metoda učenja pretraživanja. Termin \textit{rollout} originalno je
naziv za tehniku analize pozicija i poteza u igri
\textit{backgammon}.\footnote{\url{https://en.wikipedia.org/wiki/Rollout_(backgammon)}}
Tehnika je takva da krećemo dovoljan broj puta od iste pozicije i odigravajući
bacanjem kocke bilježimo rezultate. Broj pobjeda i poraza nam daje procjenu o
dobroti pozicije. U okviru podržanog učenja tehnika se koristi tako da se iz
nekog stanja nasumično odabiru sljedeći potezi -- odabir može biti baziran na
naučenoj polici. Nakon što se izvrši jedan niz poteza i dobije rezultat, cijeli
postupak se ponavlja počinjanjem iz istog stanja tako da je procjena o dobroti
pozicije što bolja.

Kako kod podržanog učenja imamo samo pristup naučenoj polici stoga termin
\textit{rollin} nije bio prisutan, ali se podrazumijeva da do željenog stanja
dolazimo koristeći naučenu policu -- ako su neka stanja nedostižna trenutnoj
naučenoj polici bilo bi bespotrebno vrednovati poteze iz stanja do kojih ne
možemo doći. U kontekstu metoda učenja pretraživanja \textit{rollin} je postupak
dolaska do određenog stanja, a kako nam je na raspolaganju naučena i referentna
polica dobro je vidjeti koja strategija vodi do brzog i konzistentnog učenja --
hoćemo li dolaziti do stanja koristeći referentnu ili naučenu policu, a možda za
svaki korak odabrati nekom vjerojatnošću jednu ili drugu. Na slici
\ref{fig:rollinout} prikazan je postupak korištenja \textit{rollin} i
\textit{rollout} polica za vrednovanje poteza gdje su potezi predstavljeni
bridovima, a stanja čvorovima.

\begin{figure}
\centering
\begin{tikzpicture}[
  >={Triangle[]},
  b/.style = {lightblue,fill=lightblue!40},
  g/.style = {darkgreen,fill=darkgreen!40}
  ]
  \matrix[
  matrix of nodes,column sep=1cm,row sep=.1cm,
  nodes={
    lightgray,draw,thick,fill=lightgray!40,circle,
    minimum size=3ex, inner sep=1pt,anchor=south
  }] (m) {    &    &    &    &     {}&     {}\\    &    &    &    &     {}&     {}\\    &     {}&     {}&|[g]|{}&|[g]|{}&|[g]|E \\    &    &    &    &     {}&     {}\\
   |[b]|S    &|[b]|{}&|[b]|R    &|[b]|{}&|[b]|{}&|[b]|E \\    &    &    &    &     {}&     {}\\    &     {}&     {}&|[g]|{}&|[g]|E    &       \\    &    &    &    &     {}&       \\    &    &    &    &     {}&       \\
  };
    % Green arrows
  \draw[darkgreen,->, line width=1.2pt] (m-5-3.east) to[bend left] (m-3-4.west);
  \draw[darkgreen,->, line width=1.2pt] (m-5-3.east) to[bend right] (m-7-4.west);
  % Gray arrows
  \foreach \i[evaluate=\i as \j using int(\i+1)] in {1,2} {
    \foreach \row/\bend in {3/left, 7/right}
      \draw[lightgray,->, line width=1.2pt] (m-5-\i.east) to[bend \bend]  (m-\row-\j.west);
  }
  \foreach \i[evaluate=\i as \j using int(\i+1)] in {4,5} {
    \foreach \row/\bend in {1/left, 2/left}
      \draw[lightgray,->, line width=1.2pt] (m-3-\i.east) to[bend \bend]  (m-\row-\j.west);
  }
  \foreach \i[evaluate=\i as \j using int(\i+1)] in {4,5} {
    \foreach \row/\bend in {4/left, 6/right}
      \draw[lightgray,->, line width=1.2pt] (m-5-\i.east) to[bend \bend]  (m-\row-\j.west);
  }
  \foreach \row/\bend in {8/right, 9/right}
    \draw[lightgray,->, line width=1.2pt] (m-7-4.east) to[bend \bend]  (m-\row-5.west);
  % Black arrows
  \foreach \i [remember=\i as \lasti (initially 4)] in {5,6}
    \draw[->, line width=1.2pt] (m-3-\lasti.east) to (m-3-\i.west);
  \foreach \i [remember=\i as \lasti (initially 1)] in {2,...,6}
    \draw[->, line width=1.2pt] (m-5-\lasti.east) to (m-5-\i.west);
  \draw[->, line width=1.2pt] (m-7-4.east) to (m-7-5.west);
  % Loss
  \node[right=of m-3-6] (loss1) {$y_e \in \mathcal{Y}, l(y_e) = 0$};
  \node[right=of m-5-6] (loss2) {$y_e \in \mathcal{Y}, l(y_e) = 1$};
  \node[right=of m-7-5] (loss3) {$y_e \in \mathcal{Y}, l(y_e) = 2$};
  \draw[->, line width=1.2pt] (m-3-6.east) to (loss1);
  \draw[->, line width=1.2pt] (m-5-6.east) to (loss2);
  \draw[->, line width=1.2pt] (m-7-5.east) to (loss3);
  % Braces
  \draw[decorate,decoration={brace,amplitude=10pt},lightblue,thick]
    (m-7-3 |- m-7-3.south) -- node[below=10pt] (rollin) {rollin} (m-5-1 |- m-7-3.south);
  \draw[decorate,decoration={brace,amplitude=10pt},lightblue,thick]
    (m-5-6 |- m-9-5.south) -- node[below=10pt] (rollout) {rollout} (m-5-4 |- m-9-5.south);
  \path (rollin) -- node[lightblue,align=right,rotate=90] {odstupanja od \\ jednog koraka} (rollout);

  \node[shape=rectangle,draw=black,thick, above=of m-5-1] (xinX) {$x \in \mathcal{X}$};
  \draw[->, line width=1.2pt,out=-90,in=120] (xinX) to (m-5-1);
\end{tikzpicture}
\caption[Prikaz postupka \textit{rollin} i \textit{rollout} kod učenja
pretraživanja.]{Prikaz postupka \textit{rollin} i \textit{rollout} kod učenja
pretraživanja. Kreće se iz stanja $S$ i odabiru se srednja odluka od moguće tri
dva puta koristeći \textit{rollin} policu. Sivi čvorovi se ne pretražuju. U
koraku $R$ algoritam učenja odabire sve moguće poteze radeći odstupanje od
jednog koraka i primijenjuje \textit{rollout} policu za svaki mogući potez
dovoljno puta da dođe do kraja. U tom trenutku dobiva se informacija o gubitku i
ona se koristi kao procjena dobrote svakog poteza. Nakon postupka moguće je
zaključiti da se gornjim potezom koji odstupa od odabranog srednjeg stanja može
gubitak smanjiti za 1.} \label{fig:rollinout}
\end{figure}

\cite{daume15lols} pokazuju da korištenje višerazrednog klasifikatora
osjetljivog na trošak koji ima dobre ograde žaljenja ne garantira konzistentnu
naučenu policu za problem strukturnog predviđanja. Ovisno o načinu na koji se
radi \textit{rollin} i \textit{rollout} i pretpostavci o tome kakva je
referentna polica moguće je dobiti nekonzistentnu redukciju. Rezultat je
priložen na slici \ref{fig:policyresult}. Ovisno o tome koje \textit{rollin} ili
\textit{rollout} police koristimo moguće je dobiti naučenu policu koja nije
konzistentna (ima veliko strukturno žaljenje) ili policu koja nije lokalno
optimalna. U slučaju korištenja naučene police za \textit{rollin} i
\textit{rollout} problem strukturnog predviđanja reducira se na problem
podržanog učenja koji je puno teži jer se za učenje uopće ne koristi znanje
referentne police. Jedini dobar pristup kojim se postiže lokalna optimalnost i
konzistentna redukcija je onaj u kojem se koristi mješovita polica za
\textit{rollout}, a opis metode lokalno optimalnog učenja pretrživanja nalazi se
u potpoglavlju \ref{ch:LOLS}. Ako se pretpostavi da je referentna polica
optimalna i to je stvarno slučaj onda će učenje koristeći \textit{rollin} s
naučenom, a \textit{rollout} s  referentnom rezultirati s konzistentnom i
lokalno optimalnom naučenom policom. Za mnoštvo problema nije lako definirati
optimalnu policu i zanimljiv je rezultat da upravo \textit{rollout} s mješovitom
policom ima garancije da će, ako je u prostoru odluka moguće raditi lokalno
uspinjanje brdom \engl{local hill climbing}, naučena polica biti konstantno
lošija od dane referentne police ili će biti bolja od referentne police. Takvu
garanciju pristupi prije \textsc{lols} algoritma nisu imali i problem je
postojao kod prve metode učenja pretraživanja -- \textsc{Searn}. Navedena
karakteristika nije pretjerano korisna ako je referentna polica jako loša, u tom
slučaju učenje se odvija kao da referenta polica ne postoji (informacija koju
ona daje je beskorisna) i u slučaju da je lako raditi uspinjanje brdom u
prostoru odluka na problemu bi jednako dobro radilo i podržano učenje.

% If you use beamer only pass "xcolor=table" option, i.e. \documentclass[xcolor=table]{beamer}
\begin{figure}
\centering
\begin{tabular}{|
>{\columncolor[HTML]{FFFFC7}}l |
>{\columncolor[HTML]{C0C0C0}}c |
>{\columncolor[HTML]{C0C0C0}}c |
>{\columncolor[HTML]{C0C0C0}}c |}
\hline
\multicolumn{1}{|c|}{\cellcolor[HTML]{C0C0C0}Rollout $\rightarrow$} & \cellcolor[HTML]{C0C0C0}                                     & \cellcolor[HTML]{C0C0C0}                                     & \cellcolor[HTML]{C0C0C0}                                   \\
\multicolumn{1}{|c|}{\cellcolor[HTML]{FFFFC7}$\downarrow$ Rollin}   & \multirow{-2}{*}{\cellcolor[HTML]{C0C0C0}\textbf{Referenca}} & \multirow{-2}{*}{\cellcolor[HTML]{C0C0C0}\textbf{Mješovita}} & \multirow{-2}{*}{\cellcolor[HTML]{C0C0C0}\textbf{Naučena}} \\ \hline
\textbf{Referenca}                                                  & \multicolumn{3}{c|}{\cellcolor[HTML]{FFCCC9}Nekonzistentna redukcija}                                                                                                                    \\ \hline
\textbf{Naučena}                                                    & \cellcolor[HTML]{FFCCC9}Nije lokalno optimalna               & \cellcolor[HTML]{C5F7C5}Dobra                                & \cellcolor[HTML]{FFCCC9}Podržano učenje                    \\ \hline
\end{tabular}
\caption[Rezultati teoretske analize načina učenja koristeći \textit{rollin} i
\textit{rollout} police.]{Rezultati teoretske analize načina učenja koristeći
\textit{rollin} i \textit{rollout} police. Ovisno o odabiru \textit{rollin} i
\textit{rollout} para naučena polica može biti nekonzistentna ili bez svojstva
lokalne optimalnosti, problem učenja police se može svesti na podržano učenje
ili naučena polica ipak može biti konzistentna i lokalno optimalna.}
\label{fig:policyresult}
\end{figure}

Da bi se izbjeglo ponavljanje nekih značajki, prednosti i ciljeva algoritama
učenja pretraživanja njihove karakteristike detaljnije su analizirane u zasebnim
potpoglavljima (\ref{ch:metodesearchlearn}).
