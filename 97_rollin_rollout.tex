\begin{figure}
\centering
\begin{tikzpicture}[
  >={Triangle[]},
  b/.style = {lightblue,fill=lightblue!40},
  g/.style = {darkgreen,fill=darkgreen!40}
  ]
  \matrix[
  matrix of nodes,column sep=1cm,row sep=.1cm,
  nodes={
    lightgray,draw,thick,fill=lightgray!40,circle,
    minimum size=3ex, inner sep=1pt,anchor=south
  }] (m) {    &    &    &    &     {}&     {}\\    &    &    &    &     {}&     {}\\    &     {}&     {}&|[g]|{}&|[g]|{}&|[g]|E \\    &    &    &    &     {}&     {}\\
   |[b]|S    &|[b]|{}&|[b]|R    &|[b]|{}&|[b]|{}&|[b]|E \\    &    &    &    &     {}&     {}\\    &     {}&     {}&|[g]|{}&|[g]|E    &       \\    &    &    &    &     {}&       \\    &    &    &    &     {}&       \\
  };
    % Green arrows
  \draw[darkgreen,->, line width=1.2pt] (m-5-3.east) to[bend left] (m-3-4.west);
  \draw[darkgreen,->, line width=1.2pt] (m-5-3.east) to[bend right] (m-7-4.west);
  % Gray arrows
  \foreach \i[evaluate=\i as \j using int(\i+1)] in {1,2} {
    \foreach \row/\bend in {3/left, 7/right}
      \draw[lightgray,->, line width=1.2pt] (m-5-\i.east) to[bend \bend]  (m-\row-\j.west);
  }
  \foreach \i[evaluate=\i as \j using int(\i+1)] in {4,5} {
    \foreach \row/\bend in {1/left, 2/left}
      \draw[lightgray,->, line width=1.2pt] (m-3-\i.east) to[bend \bend]  (m-\row-\j.west);
  }
  \foreach \i[evaluate=\i as \j using int(\i+1)] in {4,5} {
    \foreach \row/\bend in {4/left, 6/right}
      \draw[lightgray,->, line width=1.2pt] (m-5-\i.east) to[bend \bend]  (m-\row-\j.west);
  }
  \foreach \row/\bend in {8/right, 9/right}
    \draw[lightgray,->, line width=1.2pt] (m-7-4.east) to[bend \bend]  (m-\row-5.west);
  % Black arrows
  \foreach \i [remember=\i as \lasti (initially 4)] in {5,6}
    \draw[->, line width=1.2pt] (m-3-\lasti.east) to (m-3-\i.west);
  \foreach \i [remember=\i as \lasti (initially 1)] in {2,...,6}
    \draw[->, line width=1.2pt] (m-5-\lasti.east) to (m-5-\i.west);
  \draw[->, line width=1.2pt] (m-7-4.east) to (m-7-5.west);
  % Loss
  \node[right=of m-3-6] (loss1) {$y_e \in \mathcal{Y}, l(y_e) = 0$};
  \node[right=of m-5-6] (loss2) {$y_e \in \mathcal{Y}, l(y_e) = 1$};
  \node[right=of m-7-5] (loss3) {$y_e \in \mathcal{Y}, l(y_e) = 2$};
  \draw[->, line width=1.2pt] (m-3-6.east) to (loss1);
  \draw[->, line width=1.2pt] (m-5-6.east) to (loss2);
  \draw[->, line width=1.2pt] (m-7-5.east) to (loss3);
  % Braces
  \draw[decorate,decoration={brace,amplitude=10pt},lightblue,thick]
    (m-7-3 |- m-7-3.south) -- node[below=10pt] (rollin) {rollin} (m-5-1 |- m-7-3.south);
  \draw[decorate,decoration={brace,amplitude=10pt},lightblue,thick]
    (m-5-6 |- m-9-5.south) -- node[below=10pt] (rollout) {rollout} (m-5-4 |- m-9-5.south);
  \path (rollin) -- node[lightblue,align=right,rotate=90] {odstupanja od \\ jednog koraka} (rollout);

  \node[shape=rectangle,draw=black,thick, above=of m-5-1] (xinX) {$x \in \mathcal{X}$};
  \draw[->, line width=1.2pt,out=-90,in=120] (xinX) to (m-5-1);
\end{tikzpicture}
\caption{Prikaz postupka \textit{rollin} i \textit{rollout} kod učenja pretraživanja.} \label{rollinoutfig}
\end{figure}

\cite{daume15lols}

% If you use beamer only pass "xcolor=table" option, i.e. \documentclass[xcolor=table]{beamer}
\begin{figure}[]
\centering
\label{my-label}
\begin{tabular}{|
>{\columncolor[HTML]{FFFFC7}}l |
>{\columncolor[HTML]{C0C0C0}}c |
>{\columncolor[HTML]{C0C0C0}}c |
>{\columncolor[HTML]{C0C0C0}}c |}
\hline
\multicolumn{1}{|c|}{\cellcolor[HTML]{C0C0C0}Rollin $\rightarrow$} & \cellcolor[HTML]{C0C0C0}                                     & \cellcolor[HTML]{C0C0C0}                                   & \cellcolor[HTML]{C0C0C0}                                   \\
\multicolumn{1}{|c|}{\cellcolor[HTML]{FFFFC7}$\downarrow$ Rollout} & \multirow{-2}{*}{\cellcolor[HTML]{C0C0C0}\textbf{Referenca}} & \multirow{-2}{*}{\cellcolor[HTML]{C0C0C0}\textbf{Mješana}} & \multirow{-2}{*}{\cellcolor[HTML]{C0C0C0}\textbf{Naučena}} \\ \hline
\textbf{Referenca}                                                 & \multicolumn{3}{c|}{\cellcolor[HTML]{FFCCC9}Nekonzistentna redukcija}                                                                                                                  \\ \hline
\textbf{Naučena}                                                   & \cellcolor[HTML]{FFCCC9}Nije lokalno optimalna               & \cellcolor[HTML]{C5F7C5}Dobra                              & \cellcolor[HTML]{FFCCC9}Podržano učenje                    \\ \hline
\end{tabular}
\caption{Rezultati teoretske analize načina učenja koristeći \textit{rollin} i
\textit{rollout} police.}
\end{figure}
