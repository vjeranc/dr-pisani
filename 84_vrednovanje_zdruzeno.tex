Koriste se iste mjere učinkovitosti za vrednovanje modela i iste značajke
opisane u prijašnjim potpoglavljima. Podatkovni skup za učenje i testiranje je
za ovaj združeni zadatak samo iz UD, a za potrebe usporedbe razdvojen je u dva
skupa koji su identični \textsc{Set} i \textsc{Wiki} u rečenicama.

\begin{table}
\centering
\caption{Rezultat označavanja vrste riječi sa združenim modelom.}
\label{table:taggingjoint}
\begin{tabular}{|l|l|l|l|l|}
\hline
Metoda             & \textsc{\textunderscript{Set}{pos}} & \textsc{\textunderscript{Wiki}{pos}} & \textsc{\textunderscript{Set}{msd}} & \textsc{\textunderscript{Wiki}{msd}} \\ \hline \hline
\textsc{Vwpos 2}   & \textbf{98.71}                      & 96.24                                & 90.22                               & 84.13                 \\
\textsc{Vwjoint 1} & 98.69                               & \textbf{97.23}                       & \textbf{92.03}                      & \textbf{85.01}        \\ \hline
\end{tabular}
\end{table}

U tablici \ref{table:taggingjoint} prikazana je učinkovitost združenog modela
samo na zadatku označavanja vrste riječi. Za usporedbu je prikazan najuspješniji
model na tom zadatku \textsc{Vwpos 2}. Očito nije lako unaprijediti označavanje
osnovnih oznaka vrste riječi, ali se vidi bitan skok u učinkovitosti kod
morfosintaktičkih deskriptora -- UD ne koristi sve atribute koje ima
\textsc{SETimes.HR}, ali učinkovitost bez tih atributa nije značajno
promijenjena. Skok je bio očekivan jer postoje neki atributi koje možemo jasnije
označiti ako nam je dana informacija o sintaksi. Kako u okviru združenog učenja
dolazi do propagacije informacije o grešci nakon ovisnosnog parsanja, onda će
zadatak označavanja vrste riječi prilagoditi svoje težine tako da izvršavanje
ovisnosnog parsanja bude točnije. Rezultati u tablici
\ref{table:depparsing:joint} potvrđuju da združeno učenje može poboljšati
rezultate na zadatku iskorištavanjem funkcije gubitka koja u ovom slučaju daje
bolju procjenu greške. Moguće je prvo trenirati samo na jednom zadatku u
kaskadi, a onda na oba u slučaju da za prvi zadatak postoji više podataka. U
prvotnom slučaju funkcija gubitka daje lošu procjenu prave greške (jer se
združeni zadatak ne izvršava), a u naknadnom učenju cilj je pretrenirani model
prilagoditi da ima dobre performanse na združenom zadatku.

\begin{table}
\centering
\caption{Rezultat ovisnosnog parsanja združenog modela.}
\label{table:depparsing:joint}
\begin{tabular}{|l|c|c|}
\hline
Metoda                 & \textsc{las}   & \textsc{uas}    \\ \hline \hline
\textsc{Vwdep}   (MSD) & 79.22          & 86.44           \\
\textsc{Vwjoint} (POS) & 75.44          & 83.91           \\
\textsc{Vwjoint} (MSD) & \textbf{81.01} & \textbf{88.02}  \\ \hline
\end{tabular}
\end{table}
