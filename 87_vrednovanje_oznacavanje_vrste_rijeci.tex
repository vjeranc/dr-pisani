Za vrednovanje učinkovitosti koristi se mjera točnosti \engl{accuracy}. Za
usporedbu se koriste rezultati dobiveni u \citep{agic2013lemmatization} koji su
dobiveni HunPos označivačem. Vrednovanje se vrši na dva testna skupa
(\textsc{SETimes.HR} i \textsc{Wiki}).\footnote{Datoteke
\textsc{set.hr.test.conll} i \textsc{wiki.hr.test.conll} prisutne na poveznici
\url{https://github.com/ffnlp/sethr}} Vjerojatno su u međuvremenu postignuti
bolji rezultati od navedenih, ali nisu zabilježeni u raspoloživoj literaturi.

U tablici \ref{table:postagging} prikazani su rezultati. Uz prijašnje rezultate
prikazana je uspješnost običnog označivača vrste riječi (\textsc{Vwpos 1} --
alg.~\ref{alg:postagging}) i označivača s više prolaza (\textsc{Vwpos 2} --
alg.~\ref{alg:postagging:multipass}). Korištene značajke su prefiksi i sufiksi
do najveće duljine od pet znakova, oblik riječi u kojem su mala i velika slova
zamijenjeni slovom \textsc{l} i \textsc{u}, dvije susjedne riječi lijevo i desno
od trenutne i duljina riječi. Donošenje trenutne odluke uvjetuje se s tri prošle
odluke. U slučaju \textsc{Vwpos 2} algoritma korištena su 3 prolaza. Kod HunPos
označivača koristi se Markovljev lanac drugog stupnja, a za rezultate na
\textsc{Wiki} testnom skupu i dodatni vanjski leksikon.

\begin{table}
\centering
\caption[Rezultat označavanja vrste riječi.]{Rezultat označavanja vrste riječi.}
\label{table:postagging}
\begin{tabular}{|l|c|c|}
\hline
Metoda             & \textsc{Set}   & \textsc{Wiki}  \\ \hline \hline
HunPos             & 97.04          & 94.62          \\
\textsc{Vwpos 1}   & 98.18          & 96.20          \\
\textsc{Vwpos 2}   & \textbf{98.71} & 96.24          \\
\textsc{Vwmsd} 1   & 98.31          & \textbf{96.57} \\
\textsc{Vwmsd} 2   & 98.23          & 96.41          \\ \hline
\end{tabular}
\end{table}

U tablici \ref{table:msdtagging} prikazani su rezultati za označavanje koristeći
morfosintaktičke deskriptore. Značajke korištene identične su kao i za prethodni
zadatak. Pristupi \textsc{Vwpos 1} i \textsc{2} koriste se tako da se svaka
morfosintaktička oznaka gleda kao zasebna oznaka vrste riječi (takvih u
korištenom skupu ima 645). Pristupi \textsc{Vwmsd} 1 i 2 gledaju
morfosintaktičke odluke kao oznake vrste riječi s atributima. U svakom pristupu
za svaku riječ donosi se niz odluka koje na kraju rezultiraju potpunim ispravnim
morfosintaktičkim deskriptorom.  Kod pristupa \textsc{Vwmsd 1} trenutna odluka
uvjetovana je svim predthodnim odlukama za dvije prošle riječi i trenutnu riječ,
a za \textsc{Vwmsd 2} korištene su sve donešene odluke na prethodne dvije i
sljedeće dvije riječi.

U oba \textsc{Vwmsd} pristupa za riječ prvo odaberemo oznaku vrste riječi (jednu
od trinaest mogućih). Nakon toga moguće je odabrati prvi atribut, ali ne bilo
koji nego baš onaj koji odgovara za odabranu oznaku. To smanjuje broj binarnih
odluka koje je potrebno izvršiti. Pretpostavimo da označavamo rečenicu od deset
riječi. Kod \textsc{Vwpos 1} pristupa moramo za svaku riječ pozvati 645 binarne
odluke (\textsc{ova}) što na kraju rezultira s ukupno 6450 odluka. Zbog
ograničenja mogućih vrijednosti atributa s obzirom na izabranu oznaku vrste
riječi broj odluka bi trebao biti manji. Kod donošenja odluke na prvoj razini za
riječ možemo izabrati jednu od 13 mogućih. Nakon toga, pretpostavljajući da
oznaka ima 6 mogućih atributa (zamjenica ima toliko) i da svaki atribut ima 7
mogućih vrijednosti (padež kod imenice) onda će se za pojedinu riječ morati
pozvati $13+6 \cdot 7 = 55$ binarnih odluka, što je ukupno 550 za cijelu
rečenicu. U usporedbi s 6450 to je puno manje, a u praksi broj odluka je još
manji iz čega slijedi da je vrijeme učenja i testiranja puno manje. U tablici
\ref{table:postagging} prikazana je uspješnost \textsc{Vwmsd} pristupa na
običnom zadatku označavanja vrste riječi. \textsc{Vwmsd} pristupi imaju bolju
generalizaciju od \textsc{Vwpos 1} pristupa, ali očito povećani broj značajki
(svaka prijašnja odluka kojom uvjetujemo trenutnu je nova značajaka) zahtjeva
više podataka za bolju generalizaciju. Moguće je da u lanacu odluka koji
potreban za formiranje jednog morfosintaktičkog deskriptora dolazi lakše do
grešaka nego u slučaju da se jedan morfosintaktički deskriptor promatra kao
jedinstvena odluka. Ovaj bi slučaj upućivao na prisutnost pristranosti odlukama
koja se može dogoditi ako klasifikator nije dobro naučen. Razlog lošijim
rezultatima na testnom skupu \textsc{Wiki} je taj što skup sadrži više neviđenih
riječi i nepoznate morfosintaktičke deskriptore. Zbog toga pristupi koji donose
odluke po atributima imaju bolju generalizaciju na tom skupu. Ako se kreiraju
novi skup za treniranje i testiranje koji koriste sve rečenice iz svih skupova,
onda \textsc{Vwpos 2} ima najbolju točnost.

\begin{table}
\centering
\caption{Rezultat označavanja vrste riječi koristeći morfosintaktičke
deskriptore.}
\label{table:msdtagging}
\begin{tabular}{|l|c|c|}
\hline
Metoda             & \textsc{Set}   & \textsc{Wiki}  \\ \hline \hline
HunPos             & 87.11          & 80.83          \\
\textsc{Vwpos 1}   & 89.94          & 83.27          \\
\textsc{Vwpos 2}   & \textbf{90.22} & \textbf{84.13} \\
\textsc{Vwmsd 1}   & 89.19          & 82.22          \\
\textsc{Vwmsd 2}   & 89.06          & 81.90          \\ \hline
\end{tabular}
\end{table}
