Markovljeve mreže maksimalne margine \engla{maximum margin Markov
networks}{\mmmm{}} dopuštaju formulaciju problema strukturnog predviđanja kroz
problem kvadratnog programiranja \engl{quadratic programming} koristeći
formalizam stroja potpornih vektora za binarnu klasifikaciju. Formulacija
modela \mmmm{} dana je u nastavku:

\begin{equation}\label{eq:mmmm}
\begin{aligned}
  \min_{\mathbf{w}} & \quad \frac{1}{2} {\lVert\mathbf{w}\lVert}^2 + C \sum_{n=1}^{N}\sum_{\hat{y}} \xi_{n,\hat{y}} \text{ ,}              & \\
  \text{uz uvjete}  & \quad \mathbf{w}^\top \mathbf{\Phi}(x_n, y_n) - \mathbf{w}^\top \mathbf{\Phi}(x_n, \hat{y}) \ge l(x_n, y_n, \hat{y}) - \xi_{n,\hat{y}} & \quad \forall n, \forall \hat{y} \in \mathcal{Y} \text{ ,}\\
                    & \quad \xi_{n,\hat{y}} \ge 0                                                                                          & \quad \forall n, \forall \hat{y} \in \mathcal{Y} \text{ .}
\end{aligned}
\end{equation}

\noindent
Bitno je naglasiti da formulacija \mmmm{} ima previše uvjeta. Za svaki primjer
$(x_n, y_n)$ nad kojim se uči i za svaki netočni izlaz $\hat{y}$ potrebno je
izgraditi uvjet. Moguće je uz određene pretpostavke o $\mathcal{Y}$ i
$\mathbf{\Phi}$ zamijeniti eksponencijalan broj uvjeta polinomijalnim. Zamjena
je moguća jedino u slučaju problema strukturnog predviđanja kod kojeg funkcija
gubitka ima dekompoziciju preko niza -- u praksi je to uvijek Hammingov gubitak.

Za efikasno učenje i zaključivanje funkcija gubitka mora imati dekompoziciju
preko niza odluka. Prethodan uvjet ograničava \mmmm{}, ali postoje dvije
prednosti nad uvjetnim slučajnim poljima. \mmmm{} ne zahtijeva efikasan izračun
normalizacijske particijske funkcije \ref{eq:crf} i može se primijeniti na
probleme koji imaju drugačiju funkciju gubitka od Hammingovog gubitka. U praksi je
moguće primijeniti \mmmm{} samo na probleme koji aproksimiraju Hammingov gubitak
gubitkom zglobnice \engl{hinge loss}.

Čitatelja se upućuje na \citep{taskar2003maximum} za detaljniji pregled. S
obzirom na to da je velike margine moguće aproksimirati jednostavnim tehnikama
učenja pomoću gradijenta, \mmmm{} se ne primjenjuje u svojoj originalnoj
formulaciji \citep{daume2005learning, ratliff2006maximum}.
