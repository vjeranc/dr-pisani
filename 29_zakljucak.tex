Metode učenja pretraživanja relativno su nov pristup za rješavanje problema
združenog učenja i predviđanja. U zadnjih deset godina razvoj metoda omogućio je
primjenu na zadatke koji se prije nisu mogli efikasno rješavati ni jednom od
uobičajenih metodama strojnog učenja za strukturno predviđanje. Okvir učenja
pretraživanja omogućava istraživačima da vrijeme troše na razrađivanje
konceptualnih dijelova rješenja, umjesto da to vrijeme troše na ispravljanje
grešaka u izvornom kodu algoritma za učenje ili zaključivanje.

U okviru ovog diplomskog rada razvijeni su i vrednovani označivači vrste riječi
i ovisnosni parseri s razinama točnosti koje su usporedive s najboljim
rezultatima za hrvatski jezik koji su prisutni u literaturi. Najbolji model
postignut je koristeći združeno učenje i predviđanje na oba zadatka. Ovo
potvrđuje da za poboljšanje uspješnosti na pojedinačnim zadacima bi ubuduće
trebalo, ako je to moguće, zadatke rješavati združeno. Informacija koju zadaci
dijele u svim uobičajenim pristupima nije iskorištena, a ako ju želimo
iskoristiti, onda bi vjerojatno trebali prihvatiti povećanu vremensku složenost
ili koristiti računalo s boljim karakteristikama. Kako u okviru učenja
pretraživanja nema nedostataka u združenom učenju i predviđanju (vremenska
složenost se ne povećava), šteta bi bilo ne iskoristiti bolju združenu procjenu
gubitka.

U literaturi trenutno nije prisutna široka primjena učenja pretraživanja na
najteže probleme u obradi prirodnog jezika -- strojno prevođenje, sažimanje
teksta i ostalih problema koji vode do mogućnosti računala da stvarno razumije
prirodni jezik, a informacija o strukturi bi sigurno za te zadatke bila korisna.
Navedeni problemi zahtijevaju kvalitetna rješenja za osnovne zadatke
(označavanje vrste riječi, ovisnosno parsiranje), a u okviru učenja
pretraživanja moguće je sve te zadatke vršiti združeno. Mogućnost zamjene
osnovnog klasifikatora nekim modelom iz područja dubokog učenja \engl{deep
learning} i uspješnost metoda učenja pretraživanja na zadacima strukturnog
učenja, u obradi prirodnog jezika još nije razrađena i to bi mogao biti sljedeći
korak primjene. U svakom slučaju, metode učenja pretraživanja omogućavaju
rješavanje problema koji su prije bili nedostižni starijim metodama i ogromna su
prilika za istraživanje i primjenu.
