Prepoznavanje imenovanih entiteta \engl{named entity recognition} jedan je od
osnovnih problema u obradi prirodnog jezika. Cilj je u tekstu prepoznati
imenovane entitete. Problem je ilustriran na slici
\ref{fig:namedentityrecognition}. Rečenicu je potrebno segmentirati na riječi
koje bi mogle sadržavati imenovane entitete i nakon toga te segmente prepoznati.
Pristup koji se najčešće uzima je redukcija označavanja niza na prepoznavanje
imenovanih entiteta. Koristeći oznake za početak imena \textsc{b}, nastavak
imena \textsc{i} i oznaku da riječ nije ime \textsc{o} problem je jednostavno
sveden na označavanje niza. Moguće je prvo samo označiti riječi entiteta, a
kasnije prepoznati koji je tip, ali združeno prepoznavanje u obliku različitih
oznaka za različite entitete (\textsc{b-per}, \textsc{b-loc} i dr.) ima bolju
učinkovitost. Treba napomenuti da nastavak imenovanog entiteta uvijek prethodi
nastavak ili početak imenovanog entiteta. To znači da prije nego što smo
iskoristili oznaku \textsc{b} ne bi trebalo biti moguće koristiti oznaku
\textsc{i}. Ova činjenica upućuje da je ipak potrebno prilagoditi označivač ako
bi htjeli konzistentan izlaz. Drugi pristup je segmentirati i prepoznavati
entitete istovremeno. \citet{sarawagi2004semi} opisuju prilagođen model
prepoznavanja entiteta koji, umjesto da za entitet od tri riječi donese tri
zasebne odluke (\textsc{b}, \textsc{i} i \textsc{i}), istovremeno odlučuje o
duljini entiteta i njegovoj vrsti. Eksperimentalno potvrđuju da je taj pristup
bolji od prvog navedenog.

\begin{figure}[H]
\centering
\begin{dependency}
\begin{deptext}
  \textsc{b-per} \& \textsc{i-per} \& \textsc{o} \& \textsc{o} \& \textsc{o} \& \textsc{o} \& \textsc{b-loc} \& \textsc{o} \\
  Patrik         \& Baboumian      \& je         \& najjači    \& čovjek     \& u          \& Njemačkoj      \& .          \\
\end{deptext}
\end{dependency}
\caption[Rečenica s označenim imenovanim entitetima.]{Rečenica s označenim
imenovanim entitetima. Oznake koje počinju sa \textsc{b} se odnose na početak
imenovanog entiteta, a one koje počinju s \textsc{i} kao nastavak. Oznaka
\textsc{o} je za značke koje nisu imenovani entitet. Moguće je imati oznake za
osobna imena, imena privatnih organizacija, gradova, mjesta i dr.}
\label{fig:namedentityrecognition}
\end{figure}

Algoritam \ref{alg:namedentityrecognition:seq} pokazuje kako bi izgledala
implementacija prepoznavanja imenovanih entiteta koristeći okvir učenja
pretraživanja. Funkcija \textsc{Predict} prima skup mogućih oznaka za trenutnu
riječ radi održavanja konzistentnosti. Kao i kod označavanja niza, vremenska
složenost zaključivanja je $O(T M k)$ što je poboljšanje s obzirom na vremensku
složenost vjerojatnosnih grafičkih modela. Algoritam
\ref{alg:namedentityrecognition:segmentation} pokazuje kako bi izgledala
implementacija modela koji je ekvivalentan modelu opisanom u
\citep{sarawagi2004semi}. Kod ovog modela nije potrebno ponuditi niz mogućih
oznaka, nego je za odluku potrebno ponuditi dovoljno značajki (buduće riječi)
kako bi odluke o entitetima koji imaju više od jedne riječi bile ispravne. Ako
nas interesiraju samo lokacije duljine do tri riječi, onda skup oznaka može biti
\{\textsc{loc1}, \textsc{loc2}, \textsc{loc3}, \textsc{o}\}. Ako je ulaz
rečenica, onda je potrebno samo preskočiti riječi za koje je već donesena
odluka.

Sličan pristup kao u algoritmu \ref{alg:namedentityrecognition:seq} mogli bi
iskoristiti za označavanje vrste riječi s morfosintaktičkim deskriptorom (oznaka
vrste riječi koja uključuje atribute poput padeža, lica, roda i dr.). Umjesto da
se cijeli deskriptor smatra potpunom oznakom može se uvjetovati skup mogućih
oznaka za trenutno donošenje odluke (poziv funkcije \textsc{Predict}). Tako u
slučaju da je riječ u prvom prolazu označena kao imenica, onda je za drugi
prolaz potrebno ponuditi samo odluke vezane uz rod imenice (muški, ženski ili
srednji). Ovakva fleksibilnost čini bilo koji zadatak s dodatnim uvjetima
\engl{constraints} rješivim \citep{chang2012structured}.

\begin{algorithm}[H]
\caption{Prepoznavanje imenovanih entiteta.}
\label{alg:namedentityrecognition:seq}
\begin{algorithmic}[1]
\Function{\textsc{Run}}{riječi}
\State $\textit{izlaz} \gets \text{[]}$
\For{$n \gets 1$ \textbf{do} \Call{Len}{riječi}}
  \State \textit{ref} $\gets$ \text{riječi[n].točna\_oznaka}
  \State \textit{skupOznaka} $\gets \{B,O\}$
  \If{izlaz[n-1] = $B$ ili izlaz[n-1] = $I$}
    \State skupOznaka += $\{I\}$
  \EndIf
  \State \textit{izlaz[n]} $\gets$ \Call{Predict}{riječi[n], ref, skupOznaka}
\EndFor
\State \Call{Gubitak}{\# izlaz[n] $\neq$ riječi[n].točna\_oznaka}
\State \textbf{return} izlaz
\EndFunction
\end{algorithmic}
\end{algorithm}

\begin{algorithm}[H]
\caption{Prepoznavanje imenovanih entiteta sa segmentacijom.}
\label{alg:namedentityrecognition:segmentation}
\begin{algorithmic}[1]
\Function{\textsc{Run}}{riječi}
\State $\textit{izlaz} \gets \text{[]}$
\State $\textit{prethodna} \gets 0$
\For{$n \gets 1$ \textbf{do} \Call{Len}{riječi}}
  \If{(prethodna -= 1) $> 0$}
    \State \textbf{nastavi} \Comment{continue}
  \EndIf
  \State \textit{ref} $\gets$ \text{riječi[n].točna\_oznaka}
  \State \textit{izlaz[n]} $\gets$ \Call{Predict}{riječi[n], ref}
  \State prethodna = izlaz[n].duljina
\EndFor
\State \Call{Gubitak}{\# izlaz[n] $\neq$ riječi[n].točna\_oznaka}
\State \textbf{return} izlaz
\EndFunction
\end{algorithmic}
\end{algorithm}
