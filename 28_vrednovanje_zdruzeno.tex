U nastavku je prisutno vrednovanje združenog zadatka označavanja vrste riječi i
ovisnosnog parsanja. Model se uči na združenom gubitku, a ne odvojeno po
zadatku. Prvo se označavaju vrste riječi gdje se odluke propagiraju za zadatak
ovisnosnog parsanja. Nakon što je parsanje gotovo, gubitak na cijelom zadatku
propagira se i do odluka za oznake vrste riječi. Moguće je model učiti združeno
tako da odluka \textsc{Pomak} kod parsera istovremeno daje i vrstu riječi, ali
to nije izvedeno u ovom radu. Koriste se iste mjere učinkovitosti za vrednovanje
modela i iste značajke opisane u prijašnjim potpoglavljima. Podatkovni skup za
učenje i testiranje je za ovaj združeni zadatak samo iz UD, a za potrebe
usporedbe testni je skup razdvojen u dva skupa koji su identični \textsc{Set} i
\textsc{Wiki} u rečenicama.

\begin{table}
\centering
\caption[Rezultat označavanja vrste riječi sa združenim modelom.]{Rezultat
označavanja vrste riječi sa združenim modelom. Prikazana je točnost modela na
dva testna skupa.}
\label{table:taggingjoint}
\begin{tabular}{|l|l|l|l|l|}
\hline
Metoda             & \textsc{\textunderscript{Set}{pos}} & \textsc{\textunderscript{Wiki}{pos}} & \textsc{\textunderscript{Set}{msd}} & \textsc{\textunderscript{Wiki}{msd}} \\ \hline \hline
\textsc{Vwpos-2}   & \textbf{98.71}                      & 96.24                                & 90.22                               & 84.13                 \\
\textsc{Vwjoint 1} & 98.69                               & \textbf{97.23}                       & \textbf{92.03}                      & \textbf{85.01}        \\ \hline
\end{tabular}
\end{table}

U tablici \ref{table:taggingjoint} prikazana je učinkovitost združenog modela
samo na zadatku označavanja vrste riječi. Za usporedbu je prikazan najuspješniji
model na tom zadatku \textsc{Vwpos-2}. Očito nije lako unaprijediti označavanje
osnovnih oznaka vrste riječi, ali se vidi bitan skok u učinkovitosti kod
morfosintaktičkih deskriptora -- UD ne koristi sve atribute koje ima
\textsc{SETimes.HR}, ali učinkovitost bez tih atributa nije značajno
promijenjena. Skok je bio očekivan jer postoje neki atributi koje je moguće
jasnije označiti ako je prisutna informacija o sintaksi. Kako u okviru združenog
učenja dolazi do propagacije informacije o grešci nakon ovisnosnog parsanja,
onda će zadatak označavanja vrste riječi prilagoditi svoje težine tako da
izvršavanje ovisnosnog parsanja bude točnije.

\begin{table}
\centering
\caption{Rezultat ovisnosnog parsanja združenog modela.}
\label{table:depparsing:joint}
\begin{tabular}{|l|c|c|}
\hline
Metoda                 & \textsc{las}   & \textsc{uas}    \\ \hline \hline
\textsc{Vwdep}   (MSD) & 79.22          & 86.44           \\
\textsc{Vwjoint} (POS) & 75.44          & 83.91           \\
\textsc{Vwjoint} (MSD) & \textbf{81.01} & \textbf{88.02}  \\ \hline
\end{tabular}
\end{table}

Rezultati u tablici \ref{table:depparsing:joint} potvrđuju da združeno učenje
može poboljšati rezultate na zadatku iskorištavanjem funkcije gubitka koja u
ovom slučaju daje bolju procjenu greške. Moguće je prvo trenirati samo na jednom
zadatku u kaskadi, a kasnije na oba, u slučaju da za prvi zadatak postoji više
podataka. U prvotnom slučaju funkcija gubitka daje lošu procjenu prave greške
(jer se združeni zadatak ne izvršava), a u naknadnom učenju cilj je pretrenirani
model prilagoditi da ima dobre performanse na združenom zadatku.

Kao što je vidljivo iz prijašnjih rezultata, dodavanjem dodatne informacije
(prelazak sa \textsc{pos} na \textsc{msd}) poboljšava uspješnost na zadacima.
Najveći skok događa se kod združenog učenja i predviđanja. Informacija u gubitku
koju nosi združen zadatak dovoljna je za poboljšanje uspješnosti na oba zadatka.
Rezultati za zadatak ovisnosnog parsanja se približavaju rezultatima na ostalim
jezicima i vjerojatno bi bilo potrebno povećati korpus za hrvatski jezik da bi
se uspješnost još više poboljšala. Rezultati na zadatku označavanja vrste riječi
bolji su kod korištenja morfosintaktičkih deskriptora jer je za neke atribute
potrebna informacija o sintaksi rečenice.
