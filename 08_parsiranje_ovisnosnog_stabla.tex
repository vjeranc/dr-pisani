Parsiranje ovisnosnog stabla  \citep{chang2015learning}.


Na slici \ref{fig:depparsing} je ilustriran problem parsiranja ovisnosnog
stabla. Za rečenicu koja ima označene vrste riječi potrebno je pronaći ovisnosno
stablo. Trivijalan način bio bi naučiti neke težine vezane uz pojedinačne veze i
pametnom enumeracijom svih mogućih stabala naći ono koje ima najveću vjerojatnost za tu rečenicu.

\begin{figure}
\centering
\begin{dependency}[theme = simple]
\begin{deptext}
  \textsc{Rgc} \& \textsc{Vmn3pf} \& \textsc{Ncfpnn} \& \textsc{Rgp} \& \textsc{Z} \& \textsc{Cs} \& \textsc{Cs} \& \textsc{Ncfpnn} \& \textsc{Vmn3pf} \& \textsc{Rgp} \& \textsc{Z} \\
  Gore         \& gore            \& gore            \& gore         \& ,          \& nego        \& što         \& gore            \& gore            \& dolje        \& .          \\
\end{deptext}
\deproot{4}{root}
\depedge{2}{1}{poss}
\depedge{4}{2}{nsubj}
\depedge{4}{3}{advmod}
\depedge{4}{5}{xcomp}
\depedge{5}{6}{dobj}
\end{dependency}
\caption[Rečenica s oznakama vrste riječi i ovisnosnim stablom.]{Rečenica s
oznakama vrste riječi i ovisnosnim stablom. Za ovu rečenicu ovo nije jedini
mogući niz oznaka i jedino moguće stablo. Oznake R, V, N, Z i C su za prilog,
glagol, imenicu, znakove interpukcije i veznike. Oznake usmjerenih bridova
ovisnosnog stabla označavaju veze o subjektu glagola ili objektu glagola
(subjekt vrši radnju opisanu glagolom na neki objekt) i sl.}
\label{fig:depparsing}
\end{figure}

\citet{cer2010parsing} daju pregled metoda parsiranja ovisnosnih stabala. Prvi
način je iskoristiti postojeće \bq{constituency}{?} parsere i iz njih dobivati
ovisnosna stabla, ali sam postupak dobivanja \bq{constituency}{?} stabla je
dugotrajan stoga je bolje razmotriti metode koje direktno izgrađuju ovisnosna
stabla.

\begin{itemize}
  \item Koristeći CYK algoritam za parsiranje vremenska složenost iznos $O(n^5)$
  gdje je $n$ duljina rečenice.

  \item Malo drugačija reprezentacija dopušta brži algoritam
  \citep{eisner1999efficient}. Vremensku složenost moguće je smanjiti na
  $O(n^4)$ i dodatno do $O(n^3)$ pretpostavljajući odvojeni korijen \engl{root}
  stabla što je česta pretpostavka u parsiranju ovisnosnog stabla.

  \item Koristeći programiranje s uvjetima \engl{constraint programming} i
  naivan pristup pokušavajući između svih parova riječi odabrati odgovarajući
  brid ovisnosnog stabla moguće je složenost spustiti na $O(n^2)$
  \citep{covington2001fundamental}.

  \item Najveći skok je otkriće postupka parsanja linearne $O(n)$ vremenske
  složenosti čiji je razvoj prisutan u \citep{nivre03efficient,
  zhang2011transition, bohnet2012transition}. Ideja je inspirirana Shift-Reduce
  parserom.\footnote{\url{https://en.wikipedia.org/wiki/Shift-reduce_parser}} U
  slučaju parsiranja usmjerenog stabla postoje tri akcije koje je potrebno
  izvršavati -- pomak \engl{shift}, redukcija na lijevo \engl{reduce left} i
  redukcija na desno \engl{reduce right}. Za svaku akciju koristi se
  klasifikacijski postupak koji kao značajke koristi trenutno nedovršeno stablo
  i riječi u rečenici. Ako želimo označiti bridove, onda je potrebno koristiti
  klasifikacijski postupak kod akcija koje rezultiraju stvaranjem bridova. U
  okviru učenja pretraživanja implementiran je takav parser
  \citep{chang2015learning}, a i trenutno najbolji parser koristi identičan
  pristup \citep{andor2016globally}.

\end{itemize}
