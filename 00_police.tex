Polica \engl{policy} termin je za naučen model nakon postupka učenja ili
referentni sustav ili algoritam koji vrši određeno ponašanje. U području
podržanog učenja jedina polica koja postoji je ona koja je naučena. U području
učenja oponašanjem \engl{imitation learning} postoji više vrsta polica. Polica
koja je naučena i referentna polica koju se pokušava oponašati. U nastavku
slijedi nekoliko definicija koje bi trebale raščistiti koncept polica.

\begin{definition}[Polica]

  Polica $h$ je distribucija preko akcija koje su uvjetovane ulazom $x$ i
  stanjem $s$.

\end{definition}

\noindent
Skoro svi pristupi u potpoglavlju \ref{ch:metodesearchlearn} zahtijevaju pristup
efikasnoj i optimalnoj polici $\pi\ssymbol{1}$. Bez takve police učenje može
biti dugotrajno ili redukcija nekonzistentna. Potrebno je definirati optimalnu
policu.

\begin{definition}[Optimalna polica]

  Za par $(x, \mathbf{c})$ iz definicije \ref{def:jointlearn} i stanje $s =
  \langle y_1, \ldots, y_t \rangle$ u prostoru pretraživanja optimalna polica
  $\pi\ssymbol{1}(x, \mathbf{c}, s)$ je $\argmin_{y_{t+1}} \min_{y_{t+2},
  \ldots, y_T} c_{\langle y_1, \ldots, y_T \rangle}$. Tj.~$\pi\ssymbol{1}$
  odabire akciju (vrijednost za $y_{t+1}$) koja minimizira cjelokupni gubitak
  pretpostavljajući da će sve ostale buduće odluke biti donesene optimalno.

\end{definition}

\noindent
Lokalno optimalno učenje pretraživanja -- \textsc{lols} (potpoglavlje
\ref{ch:LOLS}) --  ima garancije o tome kakva će naučena polica biti, a
definicija je dana ispod.

\begin{definition}[Lokalno optimalna polica]

  Naučena polica je lokalno optimalna ako promjena bilo koje prijašnje odluke ne
  može poboljšati učinkovitost naučene police.

\end{definition}

\noindent
Potrebno je znati težinu učenja lokalno optimalne police. Možda je problem
lokalne optimalnosti toliko težak da bilo koji efikasan algoritam učenja neće
moći naučiti lokalno optimalnu policu u zadovoljavajućoj količini vremena.
Teorem koji slijedi govori o tome koliko je ažuriranja potrebno napraviti da bi
naučena polica bila lokalno optimalna.

\begin{theorem} \label{th:localopt}

  Pretpostavimo dostupnost algoritma koji ažurira police koristeći odstupanje od
  jednog koraka od trenutne police. Onda postoji problem s prostorom
  pretraživanja, razred police i funkcija gubitka gdje takav algoritam mora
  napraviti $\Omega(2^T)$ ažuriranja prije nego što nauči lokalno optimalnu
  policu.

\end{theorem}

Konstrukcija problema koja dokazuje teorem \ref{th:localopt} prisutna je u
\citep{daume15lols}. Naravno, kod problema gdje je moguće napraviti
dekompoziciju prostora odluka i funkcije gubitka -- kao što je slučaj kod
označavanja vrste riječi -- ne postoji konstrukcija koja bi zahtijevala
eksponencijalno mnogo primjera. Ilustrativniji je primjer igranja šaha. Mnoštvo
metoda koristi, uz bazu odigranih partija i dobro analiziranih otvaranja,
pretraživanje tijekom igre za pronalazak najboljih poteza. Pitanje je možemo li
naučiti šah bez pretraživanja. Možemo li naučiti policu koja će biti lokalno
optimalna? Takva polica bi garantirala da nakon odigrane partije šaha ne postoji
potez koji bi mogli ispraviti i time promijeniti rezultat igre. Tj.~ako bi
pronašli takav potez i ažurirali naučenu policu ona ne bi mogla postati bolja. U
slučaju šaha očito je da bi polica trebala postati bolja. Ako polica za svako
stanje ploče može reći optimalnu vjerojatnost dobitka partije onda otkriće
boljeg poteza bi tu vjerojatnost trebalo promijeniti. Očigledno je da naučena
polica za šah može biti lokalno optimalna samo u slučaju da je odigran ogroman
broj partija. Moguće je da šah kojeg igraju ljudi nije šah kojeg bi igrala
sveznajuća računala. Potezi koje rade ljudi imaju određenu distribuciju gdje bi
učenje lokalno optimalne police koja igra protiv čovjeka zahtijevalo manje
primjera. Dobar stvaran primjer je nedavna pobjeda AlphaGo protiv ljudskog
igrača u igri Go. \citet{silver2016mastering} izgrađuju policu koja za svako
stanje vraća vjerojatnost pobjede uzevši u obzir sljedeći potez. S obzirom na
broj mogućih igara ($\approx 2\cdot10^{170}$) vrlo je teško izgraditi takvu
policu čija će procjena vjerojatnosti biti lokalno optimalna te za poboljšanje
učinkovitosti ipak koriste pretragu tijekom zaključivanja.
