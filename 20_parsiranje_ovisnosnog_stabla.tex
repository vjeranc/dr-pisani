Na slici \ref{fig:depparsing} je ilustriran problem parsiranja ovisnosnog
stabla. Za rečenicu koja ima označene vrste riječi potrebno je pronaći ovisnosno
stablo. Trivijalno rješenje uključivalo bi učenje vjerojatnosti vezanih uz
pojedinačne veze i pametnom enumeracijom svih mogućih stabala naći ono koje ima
najveću vjerojatnost za tu rečenicu.

\begin{figure}
\centering
\begin{dependency}[theme = simple]
\begin{deptext}
  \textsc{Rgc} \& \textsc{Vmn3pf} \& \textsc{Ncfpnn} \& \textsc{Rgp} \& \textsc{Z} \& \textsc{Cs} \& \textsc{Cs} \& \textsc{Ncfpnn} \& \textsc{Vmn3pf} \& \textsc{Rgp} \& \textsc{Z} \\
  Gore         \& gore            \& gore            \& gore         \& ,          \& nego        \& što         \& gore            \& gore            \& dolje        \& .          \\
\end{deptext}
\deproot{2}{root}
\depedge{2}{1}{advmod}
\depedge{2}{3}{nsub}
\depedge{2}{4}{advmod}
\depedge{4}{5}{punct}
\depedge{9}{6}{mark}
\depedge{6}{7}{mwe}
\depedge{5}{9}{acl}
\depedge{9}{8}{nsub}
\depedge{9}{10}{advmod}
\depedge{2}{11}{punct}
\end{dependency}
\caption[Rečenica s oznakama vrste riječi i ovisnosnim stablom.]{Rečenica s
oznakama vrste riječi i ovisnosnim stablom. Za ovu rečenicu ovo nije jedini
mogući niz oznaka i jedino moguće stablo. Oznake R, V, N, Z i C su za prilog,
glagol, imenicu, znakove interpunkcije i veznike. Oznake usmjerenih bridova
ovisnosnog stabla označavaju veze o subjektu glagola ili objektu glagola
(subjekt vrši radnju opisanu glagolom na neki objekt) i dr.}
\label{fig:depparsing}
\end{figure}

\citet{cer2010parsing} daju pregled metoda parsiranja ovisnosnih stabala. Prvi
način je iskoristiti postojeće \bq{constituency}{?} parsere i iz njih dobivati
ovisnosna stabla, ali sam postupak dobivanja \bq{constituency}{?} stabla je
dugotrajan stoga je bolje razmotriti metode koje direktno izgrađuju ovisnosna
stabla.

\begin{itemize}
  \item Koristeći CYK algoritam za parsiranje vremenska složenost iznosi
  $O(n^5)$ gdje je $n$ duljina rečenice;

  \item Malo drugačija reprezentacija dopušta brži algoritam
  \citep{eisner1999efficient}. Vremensku složenost moguće je smanjiti na
  $O(n^4)$ i dodatno do $O(n^3)$, pretpostavljajući odvojeni korijen \engl{root}
  stabla, što je česta pretpostavka u parsiranju ovisnosnog stabla;

  \item Koristeći programiranje s uvjetima \engl{constraint programming} i
  naivan pristup -- pokušavajući između svih parova riječi odabrati odgovarajući
  brid ovisnosnog stabla -- moguće je složenost spustiti na $O(n^2)$
  \citep{covington2001fundamental};

  \item Najveći skok je otkriće postupka parsiranja linearne $O(n + m)$
  vremenske složenosti čiji je razvoj prisutan u \citep{nivre03efficient,
  zhang2011transition, bohnet2012transition} -- $n$ je duljina rečenice, a $m$
  veličina stabla (prijašnje metode ovise o veličini stabla, ali ona nije glavni
  faktor u vremenskoj složenosti). Ideja je inspirirana tranzicijskim parserom
  \engl{transition
  parser}.\footnote{\url{https://en.wikipedia.org/wiki/Shift-reduce_parser}} U
  slučaju parsiranja stabla s usmjerenim bridovima \engl{directed tree} postoje
  tri odluke koje je potrebno izvršavati -- pomak \engl{shift}, redukcija na
  lijevo \engl{reduce left} i redukcija na desno \engl{reduce right}. Za svaku
  odluku koristi se klasifikacijski postupak koji kao značajke koristi trenutno
  nedovršeno stablo i riječi u rečenici. Ako želimo označiti bridove, onda je
  potrebno koristiti klasifikacijski postupak kod odluka koje rezultiraju
  stvaranjem bridova. U okviru učenja pretraživanja implementiran je takav
  parser \citep{chang2015learning}, a i trenutno najbolji parser koristi
  identičan pristup \citep{andor2016globally}.

\end{itemize}

Tranzicijski parser potrebno je pravilno implementirati u okviru učenja
pretraživanja. Za okvir učenja pretraživanja potrebna je funkcija gubitka i
referentna polica. Neki od pristupa ne zahtijevaju da je referentna polica
optimalna, ali prisutnost optimalne police omogućava bržu konvergenciju modela
na dobro rješenje. Pitanje je koja je sljedeća odluka optimalna ako je prije
izvršen niz odluka u kojima nisu sve odgovarale primjeru za učenje?
\citet{goldberg2013training} opisuju kako učiniti referentnu policu optimalnom s
obzirom na neke prethodne pogreške u odlučivanju. Referentna polica ima
mogućnost prilagodbe i vrlo mali vremenski trošak. \citet{andor2016globally}
argumentiraju da prisutnost optimalne police ne rješava jaku prisutnost problema
pristranosti oznakama kod parsiranja. Slika \ref{fig:shiftreduce} prikazuje kako
tranzicijski parser izgrađuje stablo. Moguće je istovremeno donositi odluke o
vrsti riječi i o vrsti usmjerenog brida za poboljšanje učinkovitosti modela. U
tom slučaju bi operacija \textsc{Pomak} mogla imati pridruženu oznaku vrste
riječi, a operacije \textsc{RedukcijaDesno} i \textsc{RedukcijaLijevo} vrstu
veze. Naravno, postupak je moguće vršiti združeno tako da se prvo označe vrste
riječi, a nakon toga stablo, ali moguće je da prepletenost odluka na sve zadatke
dovodi do brže konvergencije.

\begin{figure}
  \tiny
  \begin{tabular}{l|rlm{4cm}}
  \hline
  \textbf{Akcija}           & \textbf{Stog}                                       & \textbf{Međuspremink}                                                       & \textbf{Lukovi} \\ \hline
                            & {[}\textbf{K}{]}                                    & {[}Gore gore$_1$ gore$_2$ gore$_3$ , nego što gore$_4$ gore$_5$ dolje .~{]} & \{\}           \\
  \textsc{Pomak}            & {[}\textbf{K} Gore{]}                               & {[}gore$_1$ gore$_2$ gore$_3$ , nego što gore$_4$ gore$_5$ dolje .~{]}      & \{\}             \\
  \textsc{RedukcijaLijevo}  & {[}\textbf{K}{]}                                    & {[}gore$_1$ gore$_2$ gore$_3$ , nego što gore$_4$ gore$_5$ dolje .~{]}      & \{(gore$_1$, Gore)\}  \\
  \textsc{Pomak}            & {[}\textbf{K} gore$_1${]}                           & {[}gore$_2$ gore$_3$ , nego što gore$_4$ gore$_5$ dolje .~{]}               & \{(gore$_1$, Gore)\}  \\
  \textsc{Pomak}            & {[}\textbf{K} gore$_1$ gore$_2${]}                  & {[}gore$_3$ , nego što gore$_4$ gore$_5$ dolje .~{]}                        & \{(gore$_1$, Gore),\}  \\
  \textsc{RedukcijaDesno}   & {[}\textbf{K} gore$_1${]}                           & {[}gore$_3$ , nego što gore$_4$ gore$_5$ dolje .~{]}                        & \{(gore$_1$, Gore), (gore$_1$, gore$_2$)\}  \\
  \textsc{Pomak}            & {[}\textbf{K} gore$_1$ gore$_3${]}                  & {[}~, nego što gore$_4$ gore$_5$ dolje .~{]}                                & \{(gore$_1$, Gore), (gore$_1$, gore$_2$)\}  \\
  \textsc{Pomak}            & {[}\textbf{K} gore$_1$ gore$_3$ , {]}               & {[} nego što gore$_4$ gore$_5$ dolje .~{]}                                  & \{(gore$_1$, Gore), (gore$_1$, gore$_2$)\}  \\
  \textsc{Pomak}            & {[}\textbf{K} gore$_1$ gore$_3$ , nego {]}          & {[}što gore$_4$ gore$_5$ dolje .~{]}                                        & \{(gore$_1$, Gore), (gore$_1$, gore$_2$)\}  \\
  \textsc{Pomak}            & {[}\textbf{K} gore$_1$ gore$_3$ , nego što{]}       & {[}gore$_4$ gore$_5$ dolje .~{]}                                            & \{(gore$_1$, Gore), (gore$_1$, gore$_2$)\}  \\
  \textsc{RedukcijaDesno}   & {[}\textbf{K} gore$_1$ gore$_3$ , nego{]}           & {[}gore$_4$ gore$_5$ dolje .~{]}                                            & \{(gore$_1$, Gore), (gore$_1$, gore$_2$), (nego, što)\}  \\
  \textsc{Pomak}            & {[}\textbf{K} gore$_1$ gore$_3$ , nego gore$_4${]}  & {[} gore$_5$ dolje .~{]}                                                    & \{(gore$_1$, Gore), (gore$_1$, gore$_2$), (nego, što)\}  \\
  \textsc{RedukcijaLijevo}  & {[}\textbf{K} gore$_1$ gore$_3$ , nego{]}           & {[} gore$_5$ dolje .~{]}                                                    & \{(gore$_1$, Gore), \ldots, (gore$_5$, gore$_4$)\}  \\
  \textsc{RedukcijaLijevo}  & {[}\textbf{K} gore$_1$ gore$_3$ , {]}               & {[} gore$_5$ dolje .~{]}                                                    & \{(gore$_1$, Gore), \ldots, (gore$_5$, nego)\}  \\
  \textsc{Pomak}            & {[}\textbf{K} gore$_1$ gore$_3$ , gore$_5$ {]}      & {[} dolje .~{]}                                                             & \{(gore$_1$, Gore), \ldots, (gore$_5$, nego)\}  \\
  \textsc{Pomak}            & {[}\textbf{K} gore$_1$ gore$_3$ , gore$_5$ dolje{]} & {[} .~{]}                                                                   & \{(gore$_1$, Gore), \ldots, (gore$_5$, nego)\}  \\
  \textsc{RedukcijaDesno}   & {[}\textbf{K} gore$_1$ gore$_3$ , gore$_5$ {]}      & {[} .~{]}                                                                   & \{(gore$_1$, Gore), \ldots, (gore$_5$, dolje)\}  \\
  \textsc{RedukcijaDesno}   & {[}\textbf{K} gore$_1$ gore$_3$ , {]}               & {[} .~{]}                                                                   & \{(gore$_1$, Gore), \ldots, (~,~, gore$_5$)\}  \\
  \textsc{RedukcijaDesno}   & {[}\textbf{K} gore$_1$ gore$_3${]}                  & {[} .~{]}                                                                   & \{(gore$_1$, Gore), \ldots, (gore$_3$,~,)\}  \\
  \textsc{RedukcijaDesno}   & {[}\textbf{K} gore$_1$ {]}                          & {[} .~{]}                                                                   & \{(gore$_1$, Gore), \ldots, (gore$_1$, gore$_3$)\}  \\
  \textsc{Pomak}            & {[}\textbf{K} gore$_1$ .~{]}                        & {[}{]}                                                                      & \{(gore$_1$, Gore), \ldots, (gore$_1$, gore$_3$)\}  \\
  \textsc{RedukcijaDesno}   & {[}\textbf{K} gore$_1$ {]}                          & {[}{]}                                                                      & \{(gore$_1$, Gore), \ldots, (gore$_1$, .~)\}  \\
  \textsc{RedukcijaDesno}   & {[}\textbf{K} {]}                                   & {[}{]}                                                                      & \{(gore$_1$, Gore), \ldots, (\textbf{K}, gore$_1$)\}  \\
  \end{tabular}
  \caption[Primjer parsiranja ovisnosnog stabla koristeći tranzicijski
  pristup.]{Primjer parsiranja ovisnosnog stabla koristeći tranzicijski pristup.
  \textbf{K} je korijen \engl{root}. Korištena je rečenica prisutna na slici
  \ref{fig:depparsing} i dobiven skup bridova odgovara prikazanom ovisnosnom
  stablu bez oznaka.}
  \label{fig:shiftreduce}
\end{figure}
