\begin{sazetak}
Strukturno predviđanje i učenje je sveprisutno u problemima obrade prirodnoga
jezika. Metode učenja pretraživanja pružaju okvir u kojem je te probleme moguće
efikasno rješavati. U ovom radu dan je pregled povijesti metoda učenja
pretraživanja i osvrt na područja u strojnom učenju koja su omogućila njihov
razvoj. Pokazana je i primjena na razne probleme u obradi prirodnog jezika.
Istaknute su i bitne razlike između ostalih pristupa strukturnom učenju koje
utvrđuju nadmoć i fleksibilnost metoda učenja pretraživanja. Ostvaren je
sustav koji zadatke označavanja vrste riječi i ovisnosnog parsanja vrši
združeno te su istaknute razlike i prednosti na pristupe koji zadatke ne gledaju
združeno. Vrednovanje sustava izvršeno je na podacima za hrvatski jezik.

\kljucnerijeci{učenje pretraživanja, obrada prirodnog jezika, strojno učenje,
združeno učenje i predviđanje, hrvatski jezik}
\end{sazetak}
