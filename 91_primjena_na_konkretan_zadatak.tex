U nastavku sljedi opis primjena metoda učenja pretraživanja na nekoliko zadataka
u obradi prirodnog jezika. Pristupi opisani u poglavlju
\ref{ch:metodesearchlearn} zahtjevaju sljedeće.

\begin{itemize}
\item \textbf{Dobru referentnu policu.} Svi pristupi -- osim \textsc{lols} --
zahtjevaju optimalnu referentnu policu inače može doći do nakupljanja pogreške
\engl{compounding error}.
\item \textbf{Dobar binarni klasifikator.} Za učenje pretraživanja potrebno je
efikasno rješenje za problem višerazredne klasifikacije osjetljive na trošak.
Koristeći težinsku \textsc{ovo} ili \textsc{ova} \engl{weighted all-pairs or
one-against-all} redukciju iz \citep{beygelzimer2005weighted,
beygelzimer2005error} u kombinaciji s \textit{costing} algoritmom iz
\citep{zadrozny2003cost} moguće je koristiti bilo koji binarni klasifikator bez
korigiranja izvornog koda.
\item \textbf{Dobro definiranu funkciju gubitka.}
\end{itemize}
